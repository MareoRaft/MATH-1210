\documentclass[12pt]{article}
 \usepackage[margin=2cm, bottom=3cm, top=2cm]{geometry}
% \usepackage[margin=1.0in]{geometry}
% \usepackage{textpos}
% \everymath{\displaystyle}

\usepackage{amsmath,amsfonts,amsthm,amssymb}
\usepackage{extramarks}
\usepackage{fancyhdr}
\usepackage{ifthen}
\usepackage{enumitem}
\usepackage{notesheets/notesheets}

\pagestyle{fancy}

\newtheoremstyle{mainstyle}
  {20pt plus 4pt minus 4pt} % Space above
  {70pt plus 4pt minus 4pt} % Space below
  {} % Body font
  {} % Indent amount
  {\bfseries} % Theorem head font
  {.} % Punctuation after theorem head
  {.5em} % Space after theorem head
  {} % Theorem head spec (can be left empty, meaning `normal')
\newtheoremstyle{longstyle}
  {20pt plus 4pt minus 4pt} % Space above
  {170pt plus 4pt minus 4pt} % Space below
  {} % Body font
  {} % Indent amount
  {\bfseries} % Theorem head font
  {.} % Punctuation after theorem head
  {.5em} % Space after theorem head
  {} % Theorem head spec (can be left empty, meaning `normal')

\theoremstyle{mainstyle} % this applies to ALL following new theorems % SEE research seminar, week 10 writeup for a possible way to exclude amsthm and [thm].


\newtheorem{thrm}[thm]{Theorem}
\newtheorem{defi}[thm]{Definition}
\newtheorem{coro}[thm]{Corollary}
\newtheorem{propo}[thm]{Proposition}
\newtheorem{lemm}[thm]{Lemma}
\newtheorem{examp}[thm]{Example}

\theoremstyle{longstyle}
\newtheorem{ex}[thm]{Challenge}



\everymath{\displaystyle}

\newcommand{\vs}{\vspace{8pt}}


\pagenumbering{gobble}
\setlength\parindent{0pt}
%
% Create Problem Sections
%
\newcommand{\enterProblemHeader}[1]{
\nobreak\extramarks{}{Problem \arabic{#1} continued on next page\ldots}\nobreak\
{}
\nobreak\extramarks{Problem \arabic{#1} (continued)}{Problem \arabic{#1} contin\
ued on next page\ldots}\nobreak{}
}
\newcommand{\exitProblemHeader}[1]{
\nobreak\extramarks{Problem \arabic{#1} (continued)}{Problem \arabic{#1} contin\
ued on next page\ldots}\nobreak{}
\stepcounter{#1}
\nobreak\extramarks{Problem \arabic{#1}}{}\nobreak{}
}

\setcounter{secnumdepth}{0}
\newcounter{partCounter}
\newcounter{homeworkProblemCounter}
\setcounter{homeworkProblemCounter}{1}
\nobreak\extramarks{Problem \arabic{homeworkProblemCounter}}{}\nobreak{}
%
% Homework Problem Environment
%
% This environment takes an optional argument. When given, it will
% adjust the
% problem counter. This is useful for when the problems given for your
% assignment aren't sequential. See the last 3 problems of this
% template for an
% example.
%                                                          ~%
%
% Homework Problem Environment
%
% This environment takes an optional argument. When given, it will adjust the
% problem counter. This is useful for when the problems given for your
% assignment aren't sequential. See the last 3 problems of this template for an
% example.
%

\newenvironment{homeworkProblem}[1][-1]{
\ifnum#1>0
\setcounter{homeworkProblemCounter}{#1}
\fi
\subsubsection{Problem \arabic{homeworkProblemCounter}}
\setcounter{partCounter}{1}
\enterProblemHeader{homeworkProblemCounter}
}{
  \vspace{35ex}
  \exitProblemHeader{homeworkProblemCounter}
}
\newcommand{\footer}{
\rfoot{\ifthenelse{\value{page}=1}{\textbf{This quiz has 2 sides}}{}}
}
\newcommand{\header}{
a\lhead{\large MATH 1210}
\chead{\large Implicit Differentiation \& 1st Deriv
  Applications Quiz} 
\rhead{\large Fall 2017}
{{Name and computing ID:} \underline{\hspace{3in}}}\\ \ \\
\fbox{
\begin{minipage}{6.5 in}
\textit{On my honor as a student, I pledge that I have neither given nor received help on this assignment.} \\ \ \\
{Signature:} {\underline {\hspace{3in}}}
\end{minipage}}
}

\everymath{\displaystyle}

%%%%%%%%%%%%%%%%%%%%%%%%%%%%%%%%%%%%%%%%%%%%%
\author{Math 1210}
\title{Compound Interest Worksheet}

\begin{document}
\maketitle
% \begin{textblock*}{100mm}(.65\textwidth,-5.3cm)
%   Name:\underline{\qquad\qquad\qquad\qquad\qquad\qquad}
% \end{textblock*}
%%%%%%%%%%%%%%%%%%%%%%%%%%%%%%%%%%%%%%%%%%%%%
As briefly discussed in class, various types of monetary investments
and debts accruse ``interest.''
\begin{defn}
  From an economics standpoint, \de{interest}
is the charge for the privilege of borrowing money, usually expressed
in terms of a percentage.
\end{defn}
Perhaps an example best illustrates what is going on.
\begin{example}
  Consider a savings account that gives \(1\%\) interest per
  year. Then, if you put \(\$100\) in the savings account, the bank
  will pay you \(1\%\) of \(\$100\) after a year. Thus, you earnings
  will be as in the table below \\ \\
  \begin{tabular}{|c|c|}
    Years after deposit&Total balance \\
    \(0\) & \(\$100.00\) \\
    \(1\) & \(\$101.00 = 100 + 0.01 \cdot 100\) \\
    \(2\) & \(\$102.01 = 101 + 0.01 \cdot 101\) \\
    \(3\) & \(\$103.03 = 102.01 + 0.01 \cdot 102.01\) \\
    \vdots & \vdots \\
    \(10\) & \(\$110.46\) \\
    \vdots & \vdots \\
    \(100\) & \(\$270.48\)
  \end{tabular}
\end{example}
In general, interest does not need to be compounded annually. Interest
can be compounded over any interval of time. It could be monthly,
daily, even every second. The general formula for compound interest is
the following: \[
  A(t) = P \left( 1+\frac{r}{n} \right)^{n\cdot t}
\]
where \(P\) is the principal (initial deposit), \(r\) is the interest
rate, and \(n\) is the number of times interest is applied per year,
and \(t\) is the amount of time in years passed since the inital
deposit. \\

In our example above, we could model the problem above by taking
\begin{align*}
  P & = \$100 \\
  r & = 0.01 \\
  n & = 1
\end{align*}
\begin{ex}
  Based on the example above, consider the function \[
    A(t) = 100 \left( 1 + \frac{0.01}{1} \right)^{1 \cdot t}
  \]
  to compute the balance of a savings account with a \(1\%\) annual
  compounded interest rate. Check that the following evaluations are
  true. (You can use a calculator.)
  \begin{enumerate}
  \item \(A(0) = 100\)
  \item \(A(1) = 101\)
  \item \(A(2) = 102.01\)
  \item \(A(3) = 103.0301\)
  \item \(A(10) = 110.462\ldots\)
  \item \(A(100) = 270.481\ldots\)
  \end{enumerate}
  Can you see how the formula works based on the original example?
\end{ex}
\begin{ex}
  I forget to pay my credit card bill of \(\$400\) and I am charged
  \(20\%\) interest on my unpaid debt compounded monthly. How much
  money will I owe if I forget for \(3\) months? How much will I owe
  if I forget for \(1\) year? (Hint: remember that \(3\) months is
  \(0.25\) years.)
\end{ex}
In class, I declared that we can model ``countinuously compounded
interest'' using the formula \[
  A(t) = P e^{rt}
\]
What is continously compounded interest and where did that formula
come from?
\begin{defn}
  \de{Continuously compounded interest} is interest that is applied
  every moment. 
\end{defn}
However, how many ``moments'' are in a year? This does not make much
sense by trying to compute, but we can model such a statement using
limits. If we applied interest every moment, we would get \[
  A(t) = \lim_{n \to \infty} P \left( 1 + \frac{r}{n} \right)^{n \cdot t}
\]
However, if we try to apply limit laws, we see that we get the
expression \(1^\infty\), which is an indeterminate form, so we must be
more clever!
\begin{ex}
  \textbf{Bonus!} We seek to evaluate the limit above.
  \begin{enumerate}
  \item Consider \(A(t) = P \left( 1 + \frac{r}{n}
    \right)^{n \cdot t}\). Apply \(\ln\) to both sides of the
    equation.
    \vspace{1in}
  \item Use log rules to bring down the exponent.
    \vspace{1in}
  \item Apply \(\lim_{n \to \infty}\) to both sides.
    \vspace{1in}
  \item Substitute \(n = \frac{r}{h}\). If \(\frac{r}{h} \to \infty\),
    what must \(h\) go to?
    \vspace{1in}
  \item What does this remind you of? Remember that \(\ln(1) =
    0\). Subtract \(\frac{\ln(1)}{h}\) from your limit. Does it remind
    you of anything now?
    \vspace{1in}
  \item Remember that \(\frac{d}{dx} \ln(x) = \frac{1}{x}\). Use this
    fact to evaluate your limit.
        \vspace{1in}
  \item We now know what \(\lim_{n \to \infty} \ln(A(t))\) is equal
    to. Now, remember that \(\ln(x)\) is a continuous function, so
    \(\lim_{n \to \infty} \ln(A(t)) = \ln(\lim_{n\to\infty}
    A(t))\). Apply this fact.
        \vspace{1in}
  \item Exponentiate both sides of the equation to solve for \(A(t)\). What is \(A(t)\)?
  \end{enumerate}

\end{ex}
\end{document}