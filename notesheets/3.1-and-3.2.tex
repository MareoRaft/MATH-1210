\documentclass[12pt, a4paper]{article}
\usepackage{../notesheets}

%%%%%%%%%%%%%%%%%%%%%%%%%%%%%%%%%%%%%%%%%%%%%%%%%%
\author{Math 1210}
\title{Notesheet. Sections 3.1 and 3.2 (Derivatives, cont.)}
\date{}

\begin{document}
\maketitle
\nameline
%%%%%%%%%%%%%%%%%%%%%%%%%%%%%%%%%%%%%%%%%%%%%%%%%%
\begin{thrm}
  Using our limit laws and the definition of the derivative, we define
  the following additional rules on derivatives for functions \(f(x)\)
  and \(g(x)\), as well as real number \(c\).
  \begin{itemize}
  \item \(\frac{d}{dx}[c f(x)] = c \frac{d}{dx}[f(x)]\)
  \item \(\frac{d}{dx}[f(x) \pm g(x)] = \frac{d}{dx}[f(x)] \pm
    \frac{d}{dx}(g(x))\)
  \end{itemize}
\end{thrm}
\begin{ex}
  The demand function for JM's desk lamps is given by \[
    p(x) = -0.1x^2 - 0.4x + 35
  \]
  where \(x\) is the quantity of lamps demanded in thousands and \(p(x)\) is
  the price of a lamp in dollars. What is \(p'(x)\)? What is the rate
  of change of the unit price when the quantity demanded is \(10,000\)
  lamps \((x = 10)\)? What is the price of a lamp at that level of demand?
\end{ex}
\begin{ex}
  Let \(f(x) = x^n\) and \(g(x) = x^m\). What is
  \(\frac{d}{dx}[f(x)g(x)]\)?
\end{ex}
\begin{defi}
  We define the \de{product rule} for derivatives to be
\end{defi}
\begin{ex}
  Let \(f(x) = (5x^2+1)(2 \sqrt{x}-1)\). What is \(f'(x)\)?
\end{ex}
\begin{ex}
  Let \(f(x) = x^n\) and \(g(x) = x\). What is \(\frac{d}{dx}[(f \circ
  g)(x)]\)? What is \(\frac{d}{dx}[(g \circ f)(x)]\)?
\end{ex}
\begin{defi}
  We define the \de{chain rule} for derivatives to be
\end{defi}
\begin{ex}
  Let \(f(x)\) and \(g(x)\) be differentiable functions. What is the
  derivative of \(\frac{f(x)}{g(x)} = f(x)[g(x)]^{-1}\)? (Hint: Use
  the product rule and the chain rule.)
\end{ex}

%%%%%%%%%%%%%%%%%%%%%%%%%%%%%%%%%%%%%%%%%%%%%%%%%%
\end{document}
