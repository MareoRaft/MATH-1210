\documentclass[12pt, a4paper]{article}
\usepackage{../notesheets}

%%%%%%%%%%%%%%%%%%%%%%%%%%%%%%%%%%%%%%%%%%%%%%%%%%
\author{Math 1210}
\title{Notesheet. Section 6.3: Area and the Definite Integral}
\date{}

\begin{document}
\maketitle
\nameline
%%%%%%%%%%%%%%%%%%%%%%%%%%%%%%%%%%%%%%%%%%%%%%%%%%
% example with F'(t) is a constant (million barrels per year). And we find the original F(t) and observe it's the area!
\begin{ex}
 An oil company produces a constant rate of $1.2$ million barrels per year.  How many barrels does it produce in 4 years? How many barrels does it produce in $t$ years?
\end{ex}
\begin{thrm}[Area under Graph of a Function]
	If $f$ is a nonnegative continuous function on $[a,b]$, then the area $A$ of the region under the graph is
	$$A = \lim_{n \to \oo} \hphantom{[f(x_1) + \ldots + f(x_n)] \Delta x}$$
	where $x_1,\ldots,x_n$ are points from the $n$ subintervals of $[a,b]$ of equal width $\Delta x = \frac{b-a}{n}$.
\end{thrm}
\begin{defi}
	If $f$ is a function defined on $[a,b]$, and
	$$\hphantom{}$$
	exists for all choices of points $x_1,\ldots,x_n$ in the subintervals, then this limit is the area under the curve and it is called the \de{definite integral} and it is denoted $\int_a^b f(x) \dx$.
\end{defi}
\begin{thrm}
  If $f$ is continuous on $[a,b]$, then $\int_a^b f(x) \dx$ exists.  (We say ``$f$ is \de{integrable} on $[a,b]$.'')
\end{thrm}
\begin{ex}
  What does $\int_2^6 (x^2 + 1) \dx$ mean in terms of area?  Draw a
  picture.  Approximate the area $\int_2^6 (x^2 + 1) \dx$ by cutting
  $[2,6]$ into 4 equal intervals. Is this approximation accurate?
\end{ex}
\vspace{1in}
\begin{ex}
  What happens if the function dips down below the $x$-axis?  What is
  the area under the curve $y = 4 - x$ on the interval $[0, 5]$?
  Using the definition, what is $\int_0^5 (4 - x) \dx$?
\end{ex}
\begin{ex}
  Compute \(F(x) = \int(4-x) \dx\). What is \(F(5)-F(0)\)?
\end{ex}
% \vspace{-2in}
%%%%%%%%%%%%%%%%%%%%%%%%%%%%%%%%%%%%%%%%%%%%%%%%%%
\end{document}
