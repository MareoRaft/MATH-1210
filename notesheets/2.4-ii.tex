\documentclass[12pt, a4paper]{article}
\usepackage{../notesheets}

%%%%%%%%%%%%%%%%%%%%%%%%%%%%%%%%%%%%%%%%%%%%%%%%%%
\author{Math 1210}
\title{Notesheet. Section 2.4 Part II}
\date{}

\begin{document}
\maketitle
\nameline
%%%%%%%%%%%%%%%%%%%%%%%%%%%%%%%%%%%%%%%%%%%%%%%%%%
\begin{ex}
  Consider the function \(f(x) = 1 + \frac{1}{x^2}\). What is
  \(f(10)\)? What is \(f(100)\)? \(f(10000)\)? Is there a positive
  number \(N\) such that \(f(N) \leq 1\)?
\end{ex}
\begin{defi}
  We define the \de{limit of \(f(x)\) at infinity} to be
\end{defi}
\begin{ex}
  What is \(\lim_{x \to \infty} \left(1+\frac{1}{x^2}\right)\)? What is \(\lim_{x
  \to \infty} x\)? What is \(\lim_{x \to \infty} \frac{x+1}{4x}\)?
 Harder question, can you figure out \(\lim_{x \to \infty} \frac{x^2+1}{5x^2+3x-1}\)? (Note that
``does not exist (DNE)'' is a valid answer.)
\end{ex}
\pagebreak
\begin{ex}
  Let \(\lim_{x \to a} f(x) = L\) and \(\lim_{x \to a} g(x) = M \neq
  0\) for some \(a\) (including
  \(\infty,-\infty\)). Let \(c\) be some number. Keeping the examples above in mind, what are
  the following limits equal to in terms of \(L,M,\) and \(c\)?
  \(\lim_{x \to a}(c \cdot f(x))\), \(\lim_{x \to a}(f(x)+g(x))\),
  \(\lim_{x \to a} (f(x) \cdot g(x))\), and \(\lim_{x \to a}
  \frac{f(x)}{g(x)}\). Given a number \(b > 0\), what is \(\lim_{x \to
  a} (f(x))^b\) assuming \(L^b\) is defined?
\end{ex}
\begin{ex}
  The average cost per book in dollars incurred by TJ Publishing in
  printing \(x\) books is given by the average cost function \[
    \ov{C}(x) = 4.5 + \frac{3000}{x}
  \]
  Evaluate \(\lim_{x \to \infty} \ov{C}(x)\) and interpret the meaning
  of this limit.
\end{ex}
\begin{defi}
    What is \(\lim_{x \to 0} \frac{x}{x}\)? What is \(\lim_{x \to 0}
  \frac{x}{x^2}\)? An \de{indeterminate form} is
\end{defi}
\begin{ex}
  Evaluate \(\lim_{x \to \infty} \frac{x}{1-x}\), \(\lim_{x \to 5}
  \frac{x^2-4x-5}{x-5}\), and \(\lim_{h \to 0} \frac{(h+1)^2 - 1}{h}\).
\end{ex}
%%%%%%%%%%%%%%%%%%%%%%%%%%%%%%%%%%%%%%%%%%%%%%%%%%
\end{document}
