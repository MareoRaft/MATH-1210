\documentclass[12pt, a4paper]{article}
\usepackage{../notesheets}

%%%%%%%%%%%%%%%%%%%%%%%%%%%%%%%%%%%%%%%%%%%%%%%%%%
\author{Math 1210}
\title{Notesheet. Monomials and Power Laws}
\date{}

\begin{document}
\maketitle
\nameline
%%%%%%%%%%%%%%%%%%%%%%%%%%%%%%%%%%%%%%%%%%%%%%%%%%
\begin{defi}
  A \de{monomial} is
\end{defi}
\begin{ex}
  Find \(2^3 \cdot 2^2\). Write your final result as a power of 2. If
  \(m,n \geq 0\), what is \(x^m \cdot x^n\)? What is \(x \cdot x^m
  \cdot y^n\)?
\end{ex}
\begin{defi}
  Given our example above, what must \(x^m \cdot x^0\) be? We define
  \(x^0\) to be
\end{defi}
\begin{ex}
  Now, given what we have learned, what should \(x^m \cdot x^{-m}\) be
  equal to?
\end{ex}
\begin{defi}
  If \(m\) is a natural number and \(x\) is a non-negative number, we
  define \(x^{\frac{1}{m}}\) to be
\end{defi}
\begin{ex}
  What is \(15^2\)? What is \(81^{\frac{1}{2}}\)? What is
  \(27^{\frac{1}{3}}\)? What is
  \((7^2)^{\frac{1}{2}}\)? Based on your answer, given a positive
  number \(x\) and positive integer \(m\), what is
  \((x^m)^{\frac{1}{m}}\)?
\end{ex}
\begin{defi}
  Given the above challenge, given a positive number \(x\) and
  rational numbers \(a,b\), what must \((x^a)^b\) be equal to?
\end{defi}
\begin{ex}
  What is \(9^{\frac{3}{2}}\)? What is \((x^4y^6)^5 \cdot
  (x^{-5}y^{-2})^{3}\)?
\end{ex}
\begin{defi}
  The \de{distributive law} is
\end{defi}
\begin{ex}
  Use the distributive law to expand \((x^2+x^{-2})(x^2+x^4)\).
\end{ex}
%%%%%%%%%%%%%%%%%%%%%%%%%%%%%%%%%%%%%%%%%%%%%%%%%%
\end{document}
