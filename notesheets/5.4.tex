\documentclass[12pt, a4paper]{article}
\usepackage{../notesheets}

%%%%%%%%%%%%%%%%%%%%%%%%%%%%%%%%%%%%%%%%%%%%%%%%%%
\author{Math 1210}
\title{Notesheet. Section 5.4: Differentiation of exponential functions}
\date{}

\begin{document}
\maketitle
\nameline
%%%%%%%%%%%%%%%%%%%%%%%%%%%%%%%%%%%%%%%%%%%%%%%%%%
\begin{ex}
  Recall the definition of $e$ (what happens when you take the derivative of $e^x$?).  Using this fact, find the derivatives of the following functions.
  \begin{enumerate}
    \item $f(x) = xe^x$
    \item $g(k) = \sqrt{3 + e^k}$
  \end{enumerate}
\end{ex}
\begin{ex}
  Of course differentiation rules that apply to functions in general can be applied to exponential functions too!  Please differentiate the following functions, applying the chain rule as needed.
  \begin{enumerate}
    \item $f(t) = e^{2t}$
    \item $h(x) = e^{\ell(x)}$ (here $\ell$ is an arbitrary function).
  \end{enumerate}
\end{ex}
\pagebreak
\begin{ex}
  MOAR derivatives of exponents!!!
  \begin{enumerate}
    \item If $y = e^{3t^2 + 1}$, find $\frac{dy}{dt}$
    % \item
  \end{enumerate}
\end{ex}
\vspace{-1.1in}
\begin{ex}
  The equation to calculate continuous compounded interest is $A = Pe^{rt}$ (you can think of $A$ as a function of $t$), where $t$ is the number of years, $P$ is the principal (amount of money you started with), $r$ is the annual interest rate, and $A$ is the accumulated amount of money at the end of $t$ years.

  What is $\frac{dA}{dt}$?  What does the variable $\frac{dA}{dt}$ represent in the context of the above?
\end{ex}
\begin{ex}
  Let $b$ be a positive constant.  What is the derivative of $b^x$?  (Hint: Rewrite $b^x$ in a clever way involving the functions $e^x$ and $\log_e(x)$, then differentiate.)
\end{ex}
% \vspace{-2in}
%%%%%%%%%%%%%%%%%%%%%%%%%%%%%%%%%%%%%%%%%%%%%%%%%%
\end{document}
