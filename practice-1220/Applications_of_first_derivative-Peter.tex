\documentclass[reqno,psamsfonts]{amsart}


%-------Packages---------
\usepackage{nopageno}
\usepackage{amssymb,amsfonts}
\usepackage[all,arc]{xy}
\usepackage{enumerate}
\usepackage{mathrsfs}
\usepackage{tikz}
\usepackage{tikz-cd}
\usepackage{graphicx}
\usepackage{tkz-euclide}
\usepackage{pgfplots}

\DeclareMathOperator{\Hom}{Hom}

%--------Theorem Environments--------
%theoremstyle{plain} --- default
\newtheorem{thm}{Theorem}[section]
\newtheorem*{thmn}{Theorem}
\newtheorem{cor}[thm]{Corollary}
\newtheorem{prop}[thm]{Proposition}
\newtheorem{lem}[thm]{Lemma}
\newtheorem{conj}[thm]{Conjecture}
\newtheorem{quest}[thm]{Question}


\theoremstyle{definition}
\newtheorem{defn}[thm]{Definition}
\newtheorem{defns}[thm]{Definitions}
\newtheorem*{defnn}{Definition}
\newtheorem{con}[thm]{Construction}
\newtheorem{exmp}[thm]{Example}
\newtheorem{exmps}[thm]{Examples}
\newtheorem{notn}[thm]{Notation}
\newtheorem{notns}[thm]{Notations}
\newtheorem{addm}[thm]{Addendum}
\newtheorem{exer}[thm]{Exercise}
\newtheorem*{warnn}{Warning}

\theoremstyle{remark}
\newtheorem{rem}[thm]{Remark}
\newtheorem*{remn}{Remark}
\newtheorem{rems}[thm]{Remarks}
\newtheorem{warn}[thm]{Warning}
\newtheorem{sch}[thm]{Scholium}

\makeatletter
\let\c@equation\c@thm
\makeatother
\numberwithin{equation}{section}

\bibliographystyle{plain}

%--------Meta Data: Fill in your info------
\title{Math 1210, section 002\\Applications of the First Derivative\\Reference: Tan Section 4.1\\10/10/2017}

\begin{document}
\maketitle
\thispagestyle{empty}
\noindent

\section*{Increasing/Decreasing Functions}

\begin{defnn}
A function $f$ is increasing on an open interval $(a,b)$ if $f(x_{1})<f(x_{2})$ whenever $x_{1}$ and $x_{2}$ are in $(a,b)$ and $x_{1}<x_{2}$. 
\end{defnn}

\begin{center}
\begin{tikzpicture}
\draw (0,0)--(7,0);
\draw(.5, -.5)--(.5,5);
\draw (1,.1)--(1,-.1);  
\draw (6,.1)--(6,-.1);  
\node at (1,-.4) {$a$};
\node at (6,-.4) {$b$};
\draw[ultra thick, blue] (1,1) to [out = 10, in =200] (6,4);
\filldraw[white] (1,1) circle(2pt);  
\draw (1,1) circle(2pt);  
\filldraw[white] (6,4) circle(2pt);  
\draw (6,4) circle(2pt);  
\draw(2.5, .1)--(2.5,-.1);
\draw (4.5,.1)--(4.5,-.1); 
\draw[dashed] (2.5, 0)--(2.5,1.6);
\draw[dashed] (4.5, 0)--(4.5,3.2);
\draw[dashed] (.5,3.2)--(4.5,3.2);
\draw[dashed] (.5,1.6)--(2.5, 1.6);
\node at (2.5,-.4) {$x_{1}$};
\node at (4.5,-.4) {$x_{2}$};
\node[left] at (.5,1.6) {$f(x_{1})$};
\node[left] at (.5,3.2) {$f(x_{2})$};
\end{tikzpicture}
\end{center}

\begin{defnn}
A function $f$ is decreasing on an open interval $(a,b)$ if $f(x_{1})>f(x_{2})$ whenever $x_{1}$ and $x_{2}$ are in $(a,b)$ and $x_{1}<x_{2}$. 
\end{defnn}

\begin{center}
\begin{tikzpicture}
\draw (0,0)--(7,0);
\draw(.5, -.5)--(.5,5);
\draw (1,.1)--(1,-.1);  
\draw (6,.1)--(6,-.1);  
\node at (1,-.4) {$a$};
\node at (6,-.4) {$b$};
\draw[ultra thick, blue] (1,4) to [out = -10, in =160] (6,1);
\filldraw[white] (1,4) circle(2pt);  
\draw (1,4) circle(2pt);  
\filldraw[white] (6,1) circle(2pt);  
\draw (6,1) circle(2pt);  
\draw(2.5, .1)--(2.5,-.1);
\draw (4.5,.1)--(4.5,-.1); 
\draw[dashed] (2.5, 0)--(2.5,3.4);
\draw[dashed] (4.5, 0)--(4.5,1.8);
\draw[dashed] (.5,1.8)--(4.5,1.8);
\draw[dashed] (.5,3.4)--(2.5, 3.4);
\node at (2.5,-.4) {$x_{1}$};
\node at (4.5,-.4) {$x_{2}$};
\node[left] at (.5,1.8) {$f(x_{2})$};
\node[left] at (.5,3.4) {$f(x_{1})$};
\end{tikzpicture}
\end{center}

\begin{warnn}
Since the inequalities in the definitions above are strict, functions which are constant on an open interval are neither increasing nor decreasing on that interval.
\end{warnn}

\newpage

\begin{thmn}
\begin{enumerate}
\item[]
\item If $f'(x)>0$ for all $x$ in $(a,b)$, then $f$ is increasing on $(a,b)$. 
\item If $f'(x)<0$ for all $x$ in $(a,b)$, then $f$ is decreasing on $(a,b)$.
\item If $f'(x)=0$ for all $x$ in $(a,b)$, then $f$ is constant on $(a,b)$. 
\end{enumerate}
\end{thmn}

\begin{remn}
The geometric intuition behind $(1)$ is that if $f'(x)>0$ for all $x$ in $(a,b)$, then that means the slope of the tangent line to the graph of $f$ at $(x, f(x))$ is positive for all $x$ in $(a,b)$. One way to think about this is that if for each $x$ in $(a,b)$ you zoom in really close to the point $(x, f(x))$, then the graph will look like a positively sloped line. Since positively sloped lines are increasing functions, this helps explain why $f$ is increasing on the interval. This is not at all a rigorous proof, but hopefully it provides some further understanding for why statement $(1)$ is true. The same type of reasoning can be applied to statements $(2)$ and $(3)$. 
\end{remn}

\subsection*{Procedure for determining intervals of increase/decrease of ``nice" functions}
\begin{enumerate}
\item Compute $f'$. Then find the $x$ values where $f'(x) = 0$ or $f'$ is discontinuous. Plot these numbers on the number line. 
\\
\item In each interval between the points plotted in part (1), pick a number $c$ and plug it into $f'$. 
\begin{enumerate}
\item If $f'(c)>0$ then $f$ is increasing on the interval containing $c$. 
\item If $f'(c)<0$ then $f$ is decreasing on the interval containing $c$. 
\end{enumerate}

\end{enumerate}
\vspace{1em}
\begin{exmp}
Let $f(x) = 2x^3-3x^2-12x+4$. Find the intervals where $f$ is increasing and also find the intervals where $f$ is decreasing.
\\
\\\textbf{Solution:} $f'(x) = 6x^2-6x-12$. Therefore, 
\begin{align*}
f'(x)=0&\iff 6x^2-6x-12=0\\
&\iff 6(x^2-x-12)=0\\
&\iff 6(x-2)(x+1)=0\\
&\iff x=2\ or\ x=-1
\end{align*}
Also, since $f'$ is a polynomial it is continuous everywhere, so there are no values of $x$ where $f'$ is discontinuous. Now plot $x=2$ and $x=-1$ on the number and choose numbers in each interval in between. The numbers we will choose are $-2$ in the left most interval, $0$ in the middle interval, and $3$ in the right most interval. 
\begin{align*}
f'(-2)=6(-2)^2-6(-2)-12=24&>0\\
f'(0)=6(0)^2-6(0)-12=-12&<0\\
f'(3)=6(3)^2-6(3)-12=24&>0
\\
\end{align*}
\begin{center}
\begin{tikzpicture}
\draw[<->] (-4,0)--(4,0);
\draw (-1,-.1)--(-1,.1);
\draw (2,-.1)--(2,.1);
\node at (-1.15,-.4) {$-1$};
\node at (2,-.4) {$2$};
\node at (-2.5,.3) {$+$};
\node at (.5,.3) {$-$};
\node at (3, .3) {$+$};
\end{tikzpicture}
\end{center}
Therefore $f$ is increasing on the intervals $(-\infty, -1)$ and $(2,\infty)$ and $f$ is decreasing on the interval $(-1,2)$. 
\end{exmp}

\begin{exmp}
Let $f(x) = \dfrac{1}{x+7}+x$. Find the intervals where $f$ is increasing and also find the intervals where $f$ is decreasing. 
\\
\\\textbf{Solution:} $f'(x) = -\dfrac{1}{(x+7)^2}+1$. Therefore, 
\begin{align*}
f'(x) = 0&\iff -\dfrac{1}{(x+7)^2}+1=0\\
&\iff \frac{1}{(x+7)^2}=1\\
&\iff 1 = (x+7)^2\\
&\iff \pm 1=x+7\\
&\iff x=-6\ or\ x = -8
\end{align*}
Also, $f'$ is discontinuous at $x=-7$. In fact, $f$ itself is discontinuous at $x=-7$ because $-7$ is not even in the domain of $f$. Now plot $x=-8$, $x=-7$ and $x=-6$ on the number line and choose numbers in each interval in between. 
\begin{align*}
f'(-9)=-\frac{1}{(-9+7)^2}+1=-\frac{1}{4}+1=\frac{3}{4}&>0\\
f'(-7.5) = -\frac{1}{(-7.5+7)^2}+1=-\frac{1}{.25}+1=-4+1=-3&<0\\
f'(-6.5)=-\frac{1}{(-6.5+7)^2}+1=-\frac{1}{.25}+1=-4+1=-3&<0\\
f'(-5)=-\frac{1}{(-5+7)^2}+1=-\frac{1}{4}+1=\frac{3}{4}&>0
\end{align*}

\begin{center}
\begin{tikzpicture}
\draw[<->] (-6,0)--(4,0);
\draw (-4,-.1)--(-4,.1);
\draw (-1,-.1)--(-1,.1);
\draw (2,-.1)--(2,.1);
\node at (-1.15,-.4) {$-7$};
\node at (-4.15,-.4) {$-8$};
\node at (1.9,-.4) {$-6$};
\node at (-2.5,.3) {$-$};
\node at (.5,.3) {$-$};
\node at (3, .3) {$+$};
\node at (-5, .3) {$+$};
\end{tikzpicture}
\end{center}
Therefore, $f$ is increasing on the intervals $(-\infty, -8)$ and $(-6,\infty)$ and $f$ is decreasing on the intervals $(-8,-7)$ and $(-7,-6)$. Note: the reason we exclude $7$ and do not combine the decreasing intervals into one is because $7$ is not in the domain of $f$. 
\end{exmp}

\begin{exmp}
Let $f(x) = \frac{3}{4}x^4+2x^3+1$. Find the intervals where $f$ is increasing and also find the intervals where $f$ is decreasing.   
\\
\\\textbf{Solution:} This problem looks pretty similar to example 0.1, however as we discussed in class, there is a subtlety in the sign chart of this example that does not come up in the first example or the second example. 
\\
\\$f'(x)=3x^3+6x^2$. Therefore, 
\begin{align*}
f'(x) = 0&\iff 3x^3+6x=0\\
&\iff 3x^2(x+2)=0\\
&\iff x=0\ or\ x=-2
\end{align*}
Also, $f'$ is continuous everywhere since it is a polynomial. Therefore, there are no values of $x$ where $f'$ is discontinuous. Now plot $x=0$ and $x=-2$ on the number line and choose numbers in each interval in between. 
\begin{align*}
f'(-3) = 3(-3)^3+6(-3)^2=-81+54=-27&<0\\
f'(-1) = 3(-1)^3+6(-1)^2=-3+6=3&>0\\
f'(1) =3(1)^3+6(1)^2=9&>0
\end{align*}

\begin{center}
\begin{tikzpicture}
\draw[<->] (-4,0)--(4,0);
\draw (-1,-.1)--(-1,.1);
\draw (2,-.1)--(2,.1);
\node at (-1.15,-.4) {$-2$};
\node at (2,-.4) {$0$};
\node at (-2.5,.3) {$-$};
\node at (.5,.3) {$+$};
\node at (3, .3) {$+$};
\end{tikzpicture}
\end{center}
Unlike in example 0.1, there are two consecutive intervals where $f'$ has the same sign and unlike in example 0.2, the value $x=0$ separating the consecutive intervals where $f'$ has the same sign is in the domain of $f$. Therefore in this case we can combine those two consecutive intervals into one single interval. Thus, $f$ is increasing on the interval $(-2,\infty)$ and $f$ is decreasing on the interval $(-\infty, -2)$. FYI, I learned yesterday that on an exam we have decided to NOT mark points off if you do not combine adjacent intervals of increase or decrease in situations like this problem, i.e. if you were to write that the $f$ is increasing on the intervals $(-2,0)$ and $(0,\infty)$ then you would still get full credit. That being said, it is technically more thorough and complete to write it the way I described above. 
\\
\\Here is a graph of the function $f(x) = \frac{3}{4}x^4+2x^3+1$ so that you can get a feel for what is happening in this example:
\begin{center}
\begin{tikzpicture}
\begin{axis}[samples=500, domain=-2.7:1, axis lines = middle,axis line style = {<->}, ytick=\empty, xtick = {-2,0} ]
\addplot[thick,blue] plot (\x, {.75*\x*\x*\x*\x+2*\x*\x*\x+1});
\end{axis}
\end{tikzpicture}
\end{center}
One way to think of this is to consider riding along the graph of $f$ like a roller coaster. From $(-2,0)$ your coaster is pointing above the horizon and then at a single instance when $x=0$ your coaster is pointing parallel to the horizon then you immediately point above the horizon again instead of going back down. The roller coaster sort of faked you out. You were going up and then it felt like you were maybe going to go down, but instead you went right back up again. That is why $f'$ has the same sign on the two adjacent intervals $(-2,0)$ and $(0,\infty)$. Thus, since $0$ is in the domain of $f$, we have that $f$ is actually increasing on the whole interval $(-2,\infty)$. 
\end{exmp}

\newpage
\section*{Relative Max/Min}

\begin{defn}
A function $f$ has a relative maximum at $x=c$ if there exists an open interval $(a,b)$ containing $c$ such that $f(x)\leq f(c)$ for all $x$ in $(a,b)$. We say that $f(c)$ is the relative maximum located at $x=c$. Concisely, we can say that $(c, f(c))$ is a relative max pair.  
\end{defn}

\begin{center}
\begin{tikzpicture}
\draw (0,0)--(7,0);
\draw (.5, -.5)--(.5,5); 
\draw[ultra thick, blue] (1,1) to [out =70, in =180] (3.5,4.5) to [out =0, in =110] (6,1);
\draw(3.5, .1)--(3.5,-.1);
\node at (3.5,-.4) {$c$};
\node[left] at (.5,4.5) {$f(c)$};
\draw[dashed] (.5,4.5)--(3.5, 4.5);
\draw[dashed] (3.5, 0)--(3.5,4.5);
\draw(2.5, .1)--(2.5,-.1);
\draw(4.5, .1)--(4.5,-.1);
\node at (2.5,-.4) {$a$};
\node at (4.5,-.4) {$b$};
\end{tikzpicture}
\end{center}

\begin{rem}
Intuitively what this is saying is that the points on the graph near $(c,f(c))$ are either at the same height or below $(c,f(c))$. Also, note that $c$ must be in the domain of $f$. For example, if there were a hole at the peak of the graph above, then $f$ would not have a relative max at $c$. 
\end{rem}



\begin{defn}
A function $f$ has a relative minimum at $x=c$ if there exists an open interval $(a,b)$ containing $c$ such that $f(c)\leq f(x)$ for all $x$ in $(a,b)$. We say that $f(c)$ is the relative minimum located at $x=c$. Concisely, we can say that $(c, f(c))$ is a relative min pair.  
\end{defn}

\begin{center}
\begin{tikzpicture}
\draw (0,0)--(7,0);
\draw (.5, -.5)--(.5,5); 
\draw[ultra thick, blue] (1,3.5) to [out= -60, in =180] (3.5,1) to [out =0, in =-120] (6,3.5);
\draw(3.5, .1)--(3.5,-.1);
\node at (3.5,-.4) {$c$};
\node[left] at (.5,1) {$f(c)$};
\draw[dashed] (.5,1)--(3.5, 1);
\draw[dashed] (3.5, 0)--(3.5,1);
\draw(2.5, .1)--(2.5,-.1);
\draw(4.5, .1)--(4.5,-.1);
\node at (2.5,-.4) {$a$};
\node at (4.5,-.4) {$b$};
\end{tikzpicture}
\end{center}

\begin{rem}
Intuitively what this is saying is that locally the points on the graph near $(c,f(c))$ are either at the same height or above $(c,f(c))$. Also, note that $c$ must be in the domain of $f$. For example, if there were a hole at the low point of the graph above, then $f$ would not have a relative min at $c$. 
\end{rem}

\begin{notn}
People often say local min/max instead of relative min/max. These terms mean the same thing. 
\end{notn}
\vspace{1em}

\noindent\textbf{Technicality:} Unlike in the definitions of increasing and decreasing, in the two definitions above the inequalities are NOT strict. Therefore, this implies for example that for every real number $x$, $(x,5)$ is both a relative max pair and a relative min pair of the constant function $f(x) =5$.  It is more important to have an intuitive understanding of the definition and to not confuse yourself about these technicalities, but I wanted to mention them for the sake of completeness. 

\begin{defnn}
A critical number of a function $f$ is any number $x$ in the domain of $f$ such that $f'(x)=0$ or $f'(x)$ does not exist. 
\end{defnn}

\begin{thmn}
If $f$ is differentiable on $(a, b)$ and $f$ has a relative max or min at $c$ in $(a,b)$, then $f'(c)=0$. A less specific way to say this theorem is that if $f$ is differentiable on $(a, b)$ and $f$ has a relative max or min at $c$ in $(a,b)$, then $c$ is a critical number of $f$. 
\end{thmn}

\begin{rem}
The intuitive non-rigorous reasoning for why this theorem is true is that at high points or low points of a smooth function the tangent line is horizontal and therefore the derivative vanishes at the corresponding $x$ value. 
\end{rem}

\begin{warnn}
The converse to this theorem is NOT TRUE!!!!!! By this we mean that if $f$ is differentiable on $(a,b)$ and $c$ is a number in $(a,b)$ such that $f'(c) = 0$, then it is not necessarily true that $f$ has a relative max or min at $c$. Another way this may be phrased is that if $f$ is differentiable on $(a,b)$ and $c$ is a number in $(a,b)$ such that $f'(c) = 0$, then it is not necessarily true that $c$ is a critical number of $f$.  
\end{warnn}

\begin{exmp}
Here is the example we discussed in class showing that the converse is not true. Let $f(x) = x^3$. Then $f'(x) = 3x^2$. Therefore, $f'(0) = 0$. However $f$ does not have a local min or max at $x=0$. Here is the graph of $f$: 
\\
\\
\begin{center}
\begin{tikzpicture}
\begin{axis}[samples=500, domain=-2:2, axis lines = middle,axis line style = {<->}, ytick=\empty, xtick = \empty ]
\addplot[thick,blue] plot (\x, {\x*\x*\x});
\end{axis}
\end{tikzpicture}
\end{center}
\end{exmp}

\newpage

\begin{exmp}
Let $f$ be the function defined on the interval $(1,12)$ whose graph is given below. Find:
\begin{enumerate}
\item the intervals where $f$ is increasing.
\item the intervals where $f$ is decreasing.
\item the critical numbers of $f$.
\item the relative max pairs.
\item the relative min pairs. 
\end{enumerate}
\begin{center}
\vspace{2em}
\begin{tikzpicture}
\draw (-.5,0)--(13,0);
\draw (0,-.5)--(0,7); 
\draw[ultra thick, blue] (1,3.5) to [out =-40, in =180] (3,2) to [out =0, in =180] (5,5) to (6,3) to [out=0, in =180] (7.5, 4.5) to [out = 0, in =180 ] (9,6) to [in = 160, out = 0] (11,1)--(12, .5);
\draw(1, .1)--(1,-.1);
\node at (1,-.4) {$1$};
\node at (12,-.4) {$12$};
\draw(3, .1)--(3,-.1);
\draw(5, .1)--(5,-.1);
\draw(6, .1)--(6,-.1);
\draw(7.5, .1)--(7.5,-.1);
\draw(9, .1)--(9,-.1);
\draw(10.01, .1)--(10.01,-.1);
\draw(12, .1)--(12,-.1);
\filldraw[white] (1,3.5)circle(2pt);
\draw (1,3.5) circle(2pt);
\filldraw[white] (12,.5)circle(2pt);
\draw (12,.5) circle(2pt);
\draw[dashed] (3,0)--(3, 2);
\draw[dashed] (5,0)--(5, 5);
\draw[dashed] (6,0)--(6, 3);
\draw[dashed] (7.5,0)--(7.5, 4.5);
\draw[dashed] (9,0)--(9, 6);
\draw[dashed] (10.01,0)--(10.01, 4);
\draw[dashed] (0, 2)--(3,2);
\draw[dashed] (0,5)--(5, 5);
\draw[dashed] (0,3)--(6, 3);
\draw[dashed] (0,4.5)--(7.5, 4.5);
\draw[dashed] (0,6)--(9, 6);
\draw[dashed] (0,4)--(10.01, 4);
\node[left] at (0,2) {$2$};
\node[left] at (0,5) {$5$};
\node[left] at (0,3) {$3$};
\node[left] at (0,4.5) {$4.5$};
\node[left] at (0,6) {$6$};
\node[left] at (0,4) {$4$};
\node at (3,-.4) {$3$};
\node at (5,-.4) {$5$};
\node at (6,-.4) {$6$};
\node at (7.5,-.4) {$7.5$};
\node at (9,-.4) {$9$};
\node at (10,-.4) {$10$};
\end{tikzpicture}
\end{center}
\textbf{Solution:}
\\
\begin{enumerate}
\item $f$ is increasing on the intervals $(3,5)$ and $(6, 9)$.
\item $f$ is decreasing on the intervals $(1,3)$ and $(5,6)$ and $(9,12)$.
\item The critical numbers of $f$ are: $x= 3$ (Since $f'(3) =0$), $x= 5$ (Since $f'(5) =DNE$), $x= 6$ (Since $f'(6) =DNE$), $x= 7.5$ (Since $f'(7.5) =0$), $x= 9$ (Since $f'(9) =0$), $x= 10$ (Since $f'(10) =DNE$ because vertical tangent line).
\item  The relative max pairs are: $(5,5)$ and $(9,6)$.
\item The relative min pairs are: $(3,2)$ and $(6,3)$. 
\end{enumerate}
Note that when finding relative min/max we ignore what happens at the endpoints of the interval. When we get to absolute min and max, endpoints will become important. 
\end{exmp}
\newpage
\noindent\textbf{First Derivative Test}
\\The first derivative test is a procedure for finding the relative maximums and relative minimums of ``nice" functions. It is basically the same algorithm for finding intervals of increase and intervals of decrease with some additional steps.
\begin{enumerate}
\item Determine the critical numbers of $f$. 
\item Draw a sign chart to determine the sign of $f'$ to the left and right of each critical number. If $c$ is a critical number and the sign chart zoomed in at $c$ looks like:
\begin{enumerate}
\item 
\hspace{2em}\begin{tikzpicture}
\draw (0,0)--(2,0);
\draw (1,-.1)--(1,.1);
\node at (1,-.4) {$c$};
\node at (.5,.2) {$+$};
\node at (1.5,.2) {$-$};
\end{tikzpicture}
\hspace{2em}Then $f$ has a relative max of $f(c)$ at $x=c$.
\\
\\
\item \hspace{2em}\begin{tikzpicture}
\draw (0,0)--(2,0);
\draw (1,-.1)--(1,.1);
\node at (1,-.4) {$c$};
\node at (.5,.2) {$-$};
\node at (1.5,.2) {$+$};
\end{tikzpicture}
\hspace{2em}Then $f$ has a relative min of $f(c)$ at $x=c$.
\\
\\
\item \hspace{2em}\begin{tikzpicture}
\draw (0,0)--(2,0);
\draw (1,-.1)--(1,.1);
\node at (1,-.4) {$c$};
\node at (.5,.2) {$+$};
\node at (1.5,.2) {$+$};
\end{tikzpicture}
\hspace{2em} or\hspace{2em}\begin{tikzpicture}
\draw (0,0)--(2,0);
\draw (1,-.1)--(1,.1);
\node at (1,-.4) {$c$};
\node at (.5,.2) {$-$};
\node at (1.5,.2) {$-$};
\end{tikzpicture}
\\
\\Then $f$ has a niether a relative max nor a relative min at $c$.
\end{enumerate}
\end{enumerate}
\vspace{2em}

\noindent\textbf{Example 0.1 (continued).}
Find the relative maximum and minimum pairs of the function $f(x) = 2x^3-3x^2-12x+4$.
\\
\\\textbf{Solution:} We have already done most of the work above. From the sign chart computed earlier in example 0.1, we see from the first derivative test that $f$ has a local min at $x=2$ and has a local max at $x=-1$. The local min at $x=2$ is 
\begin{align*}
f(2) = 2(2)^3-3(2)^2-12(2)+4=-16
\end{align*}
Therefore, the only relative minimum pair is $(2,-16)$.  The local max at $x=-1$ is 
\begin{align*}
f(-1) = 2(-1)^3-3(-1)^2-12(-1)+4=11
\end{align*}
Therefore, the only relative maximum pair is $(-1,11)$. 
\\
\\
\noindent\textbf{Example 0.3 (continued).}
Find the relative maximum and minimum pairs of the function $f(x) = \frac{3}{4}x^4+2x^3+1$.
\\
\\\textbf{Solution:} Again, we have already done most of the work above. From the sign chart computed earlier in example 0.3, we see from the first derivative test that $f$ has a local min at $x=-2$ and has no local maximums. The local min at $x=-2$ is 
\begin{align*}
f(-2) = \frac{3}{4}(-2)^4+2(-2)^3+1=-3
\end{align*}
Therefore, the only relative minimum pair is $(-2,-3)$. 

\newpage

\begin{exmp}
Find the relative maximum and minimum pairs of the function $f(x) = x^{2/3}$.
\begin{align*}
f'(x) = \frac{2}{3}x^{-1/3}=\frac{2}{3x^{1/3}}
\end{align*}
In this case, $f'(x)$ never equals zero, however $f'$ does have one discontinuity at $x=0$ and $x=0$ is in the domain of $f$. Therefore, the only critical number of $f$ is $x=0$. Also, 
\begin{align*}
f'(-1) = \frac{2}{3(-1)^{1/3}}=-\frac{2}{3}&<0\\
f'(1) = \frac{2}{3(1)^{1/3}}=\frac{2}{3}&>0
\end{align*}
\begin{center}
\begin{tikzpicture}
\draw[<->] (-2,0)--(2,0);
\draw (0,-.1)--(0,.1);
\node at (0,-.4) {$0$};
\node at (-1,.4) {$-$};
\node at (1,.4) {$+$};
\end{tikzpicture}
\end{center}
Therefore, $f$ has a local min at $x=0$ and the min is $f(0) = (0)^{2/3}=0$. Thus, the only local min pair of $f$ is $(0, 0)$. 
\end{exmp}
\vspace{2em}
\noindent\textbf{Examples of some technicalities that arise in the definition of relative extrema.}
\\If you are similar to me in the sense that you like to understand definitions very thoroughly and analyze many of the different possibilities, then you may find it helpful to go through the examples below again. We talked about these in class. 

\begin{exmp}
Let $f$ be the constant function $f(x) = 3$ defined on the interval $(1,6)$. The graph of $f$ is shown below. 
\begin{center}
\begin{tikzpicture}
\draw (-.5,0)--(7,0);
\draw(0, -.5)--(0,5);
\draw (1,.1)--(1,-.1);  
\draw (6,.1)--(6,-.1);  
\node at (1,-.4) {$1$};
\node at (6,-.4) {$6$};
\draw[ultra thick, blue] (1,3) to (6,3);
\filldraw[white] (1,3) circle(2pt);  
\draw (1,3) circle(2pt);  
\filldraw[white] (6,3) circle(2pt);  
\draw (6,3) circle(2pt);  
\node[left] at (0,3) {$3$};
\end{tikzpicture}
\end{center}
\end{exmp}
Because of the non-strict inequality in the definitions of relative max and relative min, we see that $(x, 3)$ is both a relative max pair and a relative min pair for all $x$ in $(1,6)$. 

\newpage
\begin{exmp}
Let $f$ be the function defined on the interval $(1,6)$ whose graph is shown below. 
\begin{center}
\begin{tikzpicture}
\draw (-.5,0)--(7,0);
\draw(0, -.5)--(0,5);
\draw (1,.1)--(1,-.1);  
\draw (3.5,.1)--(3.5,-.1); 
\draw (6,.1)--(6,-.1);  
\node at (1,-.4) {$1$};
\node at (3.5,-.4) {$3.5$};
\node at (6,-.4) {$6$};
\filldraw (3.5,4.5) circle(2pt);  
\draw[ultra thick, blue] (1,3) to (6,3);
\filldraw[white] (1,3) circle(2pt);  
\draw (1,3) circle(2pt);  
\filldraw[white] (6,3) circle(2pt);  
\draw (6,3) circle(2pt);  
\filldraw[white] (3.5,3) circle(2pt);  
\draw (3.5,3) circle(2pt);  
\node[left] at (0,3) {$3$};
\node[left] at (0,4.5) {$4.5$};
\end{tikzpicture}
\end{center}
\end{exmp}
In this case, we see that $(x, 3)$ is both a relative max and relative min pair for all $x$ in $(1,6)$ except for $x=3.5$. Also, $(3.5, 4.5)$ is a relative max pair. However, $f$ does not have a relative min at $x=3.5$. 

\begin{exmp}
Let $f$ be the function defined on the interval $(1,6)$ whose graph is shown below. 
\begin{center}
\begin{tikzpicture}
\draw (-.5,0)--(7,0);
\draw(0, -.5)--(0,5);
\draw (1,.1)--(1,-.1);  
\draw (3.5,.1)--(3.5,-.1); 
\draw (6,.1)--(6,-.1);  
\node at (1,-.4) {$1$};
\node at (3.5,-.4) {$3.5$};
\node at (6,-.4) {$6$};
\draw[ultra thick, blue] (1,1) to (3.5,1);
\draw[ultra thick, blue] (3.5,3) to (6,3);
\filldraw[white] (1,1) circle(2pt);  
\draw (1,1) circle(2pt);  
\filldraw[white] (3.5,1) circle(2pt);  
\draw (3.5,1) circle(2pt);  
\filldraw[white] (6,3) circle(2pt);  
\filldraw (3.5,3) circle(2pt);  
\draw (6,3) circle(2pt);  
\node[left] at (0,3) {$3$};
\node[left] at (0,1) {$1$};
\end{tikzpicture}
\end{center}
\end{exmp}
In this case, $(x, 1)$ is both a relative max and relative min pair for all $x$ in $(1,3.5)$ and $(x, 3)$ is both a relative max and relative min pair for all $x$ in $(3.5,6)$. Also, $(3.5, 3)$ is a relative max pair. However, $f$ does not have a relative min at $x=3.5$.  
\newpage
\begin{exmp}
This example looks very similar to the previous one. The only difference is the location of the filled-in circle. Let $f$ be the function defined on the interval $(1,6)$ whose graph is shown below.  
\begin{center}
\begin{tikzpicture}
\draw (-.5,0)--(7,0);
\draw(0, -.5)--(0,5);
\draw (1,.1)--(1,-.1);  
\draw (3.5,.1)--(3.5,-.1); 
\draw (6,.1)--(6,-.1);  
\node at (1,-.4) {$1$};
\node at (3.5,-.4) {$3.5$};
\node at (6,-.4) {$6$};
\draw[ultra thick, blue] (1,1) to (3.5,1);
\draw[ultra thick, blue] (3.5,3) to (6,3);
\filldraw[white] (1,1) circle(2pt);  
\draw (1,1) circle(2pt);  
\filldraw[white] (3.5,3) circle(2pt);  
\draw (3.5,3) circle(2pt);  
\filldraw[white] (6,3) circle(2pt);  
\filldraw (3.5,1) circle(2pt);  
\draw (6,3) circle(2pt);  
\node[left] at (0,3) {$3$};
\node[left] at (0,1) {$1$};
\end{tikzpicture}
\end{center}
\end{exmp}
In this case, $(x, 1)$ is both a relative max and relative min pair for all $x$ in $(1,3.5)$ and $(x, 3)$ is both a relative max and relative min pair for all $x$ in $(3.5,6)$. Also, $(3.5, 1)$ is a relative min pair. However, $f$ does not have a relative max at $x=3.5$.  



\end{document}


