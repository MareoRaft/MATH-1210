\documentclass[12pt, a4paper]{article}
\usepackage{../notesheets}

%%%%%%%%%%%%%%%%%%%%%%%%%%%%%%%%%%%%%%%%%%%%%%%%%%
\author{Math 1210}
\title{Notesheet. Section 5.6: Exponential Functions as Mathematical Models}
\date{}

\begin{document}
\maketitle
\nameline
%%%%%%%%%%%%%%%%%%%%%%%%%%%%%%%%%%%%%%%%%%%%%%%%%%
\begin{defi}
Consider a quantity \(Q\) that is growing (decreasing) over time,
\(t\). Then, we say a quantity exhibits \de{exponential growth
  (decay)} if
\end{defi}
\begin{ex}
  Given that a quantity \(Q(t)\) is described by the function \[
    Q(t) = 300 e^{0.02 t}
  \]
  where \(t\) is measured in minutes, answer the following questions.
  \begin{enumerate}
  \item What is the growth constant?
  \item What quantity is present initially (that is, at \(t=0\)).
  \item Compute the following values: \(Q(10), Q(100), Q(1000)\)
  \end{enumerate}
\end{ex}
\vspace{-0.8in}
\begin{ex}
  Under ideal conditions, a certain bacteria population is known to
  double every three hours. Suppose that there are initially \(100\)
  bacteria.
  \begin{enumerate}
  \item What is the size of the population after \(15\) hours?
    \vspace{0.75in}
  \item Write down an equation for \(Q(t)\) in the form \(Q(t) = Q_0
    2^{kt}\) where \(Q(0) = 100\) and \(Q(15)\) is what you found
    above.
    \vspace{0.75in}
  \item Use your equation to estimate the size of the population after
    \(20\) hours.
    \vspace{0.75in}
  \item Using exponential and log tricks/rules, rewrite your equation in the
    form \(Q(t) = Q_0 e^{k_{\text{new}} t}\).
    \vspace{0.75in}
  \end{enumerate}
\end{ex}
\vspace{-2in}
\begin{defi}
  Radioactive substances decay exponentially. The \de{half-life of a
    radioactive substance} is
\end{defi}
\vspace{-0.5in}
\begin{ex}
  Carbon dating is the process using radioactive decay of Carbon 14 to
  determine the age of a fossil. Carbon 14 has a half-life of 5730
  years. If we model its decay by an exponential decay function \(Q(t)
  = Q_0 e^{kt}\) (remember decay means \(k < 0\)), what is its decay constant, \(k\)? After \(11460\)
  years (\(2\) half lives), how fast is Carbon 14 decaying?
\end{ex}
\vspace{1.0in}
\begin{ex}
  You deposited $\$5{,}000$ in a bank in 2013.  It gains continuously compounded interest at $3\%$ APY.  How much money will you retrieve from the bank in 2019?
\end{ex}
%%%%%%%%%%%%%%%%%%%%%%%%%%%%%%%%%%%%%%%%%%%%%%%%%%
\end{document}
