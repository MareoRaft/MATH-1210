\documentclass[12pt, a4paper]{article}
\usepackage{../notesheets}

%%%%%%%%%%%%%%%%%%%%%%%%%%%%%%%%%%%%%%%%%%%%%%%%%%
\author{Math 1210}
\title{Notesheet. Section 2.6 (Derivatives)}
\date{}

\begin{document}
\maketitle
\nameline
%%%%%%%%%%%%%%%%%%%%%%%%%%%%%%%%%%%%%%%%%%%%%%%%%%
\begin{defi}
  Let \(f(x)\) be a function. We define the \de{average rate of change} of
  \(f\) from \(a\) to \(b\) by
\end{defi}
\begin{ex}
  At time \(t = 0\), a car traveling in a straight line  at \(15\) m/s
  (roughly \(34\)mph)
  starts accelerating at \(5\) m/s/s (roughly \(11\) mph/s). The
  position of the car is modeled by \[
    x(t) = 2.5 t^2 + 15 t
  \]
  What is the average rate of change of \(x(t)\) from \(t=1\) to
  \(t=2\)? What about the average rate of change of \(x(t)\) from
  \(t=1\) to \(t= 1.0001 = (1 + 0.0001)\)?
\end{ex}
\begin{defi}
  Given \(f(x)\), we define the \de{derivative of \(f(x)\)} at \(x = a\) as
\end{defi}
\begin{ex}
  Let \(f(x) = 2.5x^2 + 15x\). Use the definition of the derivative to
  compute \(f'(1)\).
\end{ex}
\begin{ex}
  Derivatives need not always exist. Does \[
    f(x) = |x| =
    \begin{cases}
      -x & x < 0 \\
      x & x \geq 0
    \end{cases}
  \]
  have a derivative at \(x=0\)?
\end{ex}
\begin{ex}
  For a constant \(c\), compute \(\frac{d}{dx}(c)\). Does your answer
  make sense?
\end{ex}
\begin{ex}
  Using the binomial theorem \[
(x+h)^n = x^n +  n \left(h x^{n-1} +
    \frac{n-1}{2} h^2 x^{n-2} + \cdots
    \cdots + \frac{n-1}{2} h^{n-2} x^2 + h^{n-1} x\right) + h^n,
  \] compute the derivative for
  \(f(x) = x^n\) where \(n\) is a positive integer.
\end{ex}
\begin{defi}
  We define the \de{power rule} for any real number \(n\) to be \(\frac{d}{dx}(x^n) = \)
\end{defi}
%%%%%%%%%%%%%%%%%%%%%%%%%%%%%%%%%%%%%%%%%%%%%%%%%%
\end{document}
