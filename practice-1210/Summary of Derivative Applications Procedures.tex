\documentclass[reqno,psamsfonts]{amsart}


%-------Packages---------
\usepackage[margin=.5in]{geometry}
\usepackage{nopageno}
\usepackage{amssymb,amsfonts}
\usepackage[all,arc]{xy}
\usepackage{enumerate}
\usepackage{mathrsfs}
\usepackage{tikz}
\usepackage{tikz-cd}
\usepackage{graphicx}
\usepackage{tkz-euclide}

\DeclareMathOperator{\Hom}{Hom}

%--------Theorem Environments--------
%theoremstyle{plain} --- default
\newtheorem{thm}{Theorem}[section]
\newtheorem{cor}[thm]{Corollary}
\newtheorem{prop}[thm]{Proposition}
\newtheorem{lem}[thm]{Lemma}
\newtheorem{conj}[thm]{Conjecture}
\newtheorem{quest}[thm]{Question}

\theoremstyle{definition}
\newtheorem{defn}[thm]{Definition}
\newtheorem{defns}[thm]{Definitions}
\newtheorem{con}[thm]{Construction}
\newtheorem{exmp}[thm]{Example}
\newtheorem{exmps}[thm]{Examples}
\newtheorem{notn}[thm]{Notation}
\newtheorem{notns}[thm]{Notations}
\newtheorem{addm}[thm]{Addendum}
\newtheorem{exer}[thm]{Exercise}

\theoremstyle{remark}
\newtheorem{rem}[thm]{Remark}
\newtheorem{rems}[thm]{Remarks}
\newtheorem{warn}[thm]{Warning}
\newtheorem{sch}[thm]{Scholium}

\makeatletter
\let\c@equation\c@thm
\makeatother
\numberwithin{equation}{section}

\bibliographystyle{plain}

%--------Meta Data: Fill in your info------
\title{Math 1210, section 002\\Summary of Derivative Application Procedures\\10/16/2017}

\begin{document}
\maketitle
\thispagestyle{empty}
\noindent

\subsection*{Procedure for determining intervals of increase/decrease of ``nice" functions}
\begin{enumerate}
\item Compute $f'$. Then find the $x$ values where $f'(x) = 0$ or $f'$ is discontinuous. Plot these numbers on the number line. 
\\
\item In each open interval between the numbers plotted in part (1), pick a number $c$ and plug it into $f'$. 
\begin{enumerate}
\item If $f'(c)>0$, then $f$ is increasing on the corresponding open interval containing $c$. Write a plus sign above this interval.
\item If $f'(c)<0$, then $f$ is decreasing on the corresponding open interval containing $c$. Write a minus sign above this interval. 
\\
\end{enumerate}
\item Write down the intervals, in interval notation, on which $f$ is increasing. If there is more than one interval write ``and'' between rather than $\cup$. Do the same thing for the intervals on which $f$ is decreasing. 
\end{enumerate}
\subsection*{First Derivative Test: Procedure for determining relative max(s) and relative min(s) of ``nice" functions}
\begin{enumerate}
\item Determine the critical numbers of $f$. In other words, calculate $f'(x)$. Then, find the values $x$ \textbf{IN THE DOMAIN OF $f$} for which $f'(x) =0$ or $f'(x)$ DNE. Plot these numbers on the number line. Notice the difference in this step and step (1) of the previous procedure. 
\\
\item In each open interval between the numbers plotted in part (1), pick a number $c$ and plug it into $f'$. If $f'(c)>0$, put a plus sign above the corresponding interval. If $f'(c)<0$ put a minus sign above the corresponding interval. If $x$ is a critical number and the sign chart zoomed in at $x$ looks like:
\begin{enumerate}
\item 
\hspace{2em}\begin{tikzpicture}
\draw (0,0)--(2,0);
\draw (1,-.1)--(1,.1);
\node at (1,-.4) {$x$};
\node at (.5,.2) {$+$};
\node at (1.5,.2) {$-$};
\end{tikzpicture}
\hspace{2em}Then $f$ has a relative max of $f(x)$ at $x$.
\\
\\
\item \hspace{2em}\begin{tikzpicture}
\draw (0,0)--(2,0);
\draw (1,-.1)--(1,.1);
\node at (1,-.4) {$x$};
\node at (.5,.2) {$-$};
\node at (1.5,.2) {$+$};
\end{tikzpicture}
\hspace{2em}Then $f$ has a relative min of $f(x)$ at $x$.
\\
\\
\item \hspace{2em}\begin{tikzpicture}
\draw (0,0)--(2,0);
\draw (1,-.1)--(1,.1);
\node at (1,-.4) {$x$};
\node at (.5,.2) {$+$};
\node at (1.5,.2) {$+$};
\end{tikzpicture}
\hspace{2em} or\hspace{2em}\begin{tikzpicture}
\draw (0,0)--(2,0);
\draw (1,-.1)--(1,.1);
\node at (1,-.4) {$x$};
\node at (.5,.2) {$-$};
\node at (1.5,.2) {$-$};
\end{tikzpicture}
\\
\\Then $f$ has a niether a relative max nor a relative min at $x$.
\\
\end{enumerate}
\item Write down the relative max(s) and relative min(s). 
\end{enumerate}

\subsection*{Second Derivative Test: An alternate (less powerful) procedure for determining relative max(s) and min(s) of ``nice" functions}
\begin{enumerate}
\item Compute $f'$ and $f''$. Find all $x$ values where $f'(x) = 0$.
\\
\item For each value $x$ from part (1), plug $x$ into $f''$. 
\begin{enumerate}
\item If $f''(x)<0$, then $f$ has a relative max of $f(x)$ at $x$.\\
\item If $f''(x)>0$, then $f$ has a relative min of $f(x)$ at $x$.\\
\item If $f''(x) = 0$ or $f''(x)$ DNE, then we cannot conclude anything. Therefore, this test is less powerful. It will not always tell us as much as the first derivative test. 
\\
\end{enumerate}
\item Write down the relative max(s) and relative min(s). 
\end{enumerate}
\newpage


\subsection*{Procedure for determining intervals of concavity of ``nice" functions}
\begin{enumerate}
\item Compute $f''$. Then find the values of $x$ where $f''(x) = 0$ or $f''(x)$ DNE. Plot these numbers on the number line. 
\\
\item In each open interval between the numbers plotted in part (1), pick a number $c$ and plug it into $f''$. 
\begin{enumerate}
\item If $f''(c)>0$, then the graph of $f$ is concave upward on the corresponding interval containing $c$. Write a plus sign above this interval.
\item If $f''(c)<0$, then the graph of $f$ is concave downward on the corresponding interval containing $c$. Write a minus sign above this interval.
\\
\end{enumerate}
\item Write down the intervals, in interval notation, on which $f$ is concave upward. If there is more than one interval write ``and'' between rather than $\cup$. Do the same thing for intervals on which $f$ is concave down. 
\end{enumerate}
\vspace{2em}
\subsection*{Procedure for determining inflection points of ``nice" functions}
\begin{enumerate}
\item Compute $f''$. Find the values $x$ \textbf{IN THE DOMAIN OF $f$} for which $f''(x) = 0$ or $f''(x)$ DNE.
\\
\item In each open interval between the numbers plotted in part (1), pick a number $c$ and plug it into $f''$. If $f''(c)>0$, put a plus sign above the corresponding interval. If $f''(c)<0$ put a minus sign above the corresponding interval. If $x$ is a number from part (1), and the sign chart zoomed in at $x$ looks like:
\\
\begin{enumerate}
\item \hspace{2em}\begin{tikzpicture}
\draw (0,0)--(2,0);
\draw (1,-.1)--(1,.1);
\node at (1,-.4) {$x$};
\node at (.5,.2) {$+$};
\node at (1.5,.2) {$-$};
\end{tikzpicture}
\hspace{2em} or\hspace{2em}\begin{tikzpicture}
\draw (0,0)--(2,0);
\draw (1,-.1)--(1,.1);
\node at (1,-.4) {$x$};
\node at (.5,.2) {$-$};
\node at (1.5,.2) {$+$};
\end{tikzpicture}
\\
\\Then $(x,f(x))$ is an inflection point.
\\
\\
\item \hspace{2em}\begin{tikzpicture}
\draw (0,0)--(2,0);
\draw (1,-.1)--(1,.1);
\node at (1,-.4) {$x$};
\node at (.5,.2) {$+$};
\node at (1.5,.2) {$+$};
\end{tikzpicture}
\hspace{2em} or\hspace{2em}\begin{tikzpicture}
\draw (0,0)--(2,0);
\draw (1,-.1)--(1,.1);
\node at (1,-.4) {$x$};
\node at (.5,.2) {$-$};
\node at (1.5,.2) {$-$};
\end{tikzpicture}
\\
\\Then $(x, f(x))$ is not an inflection point.
\\
\end{enumerate}
\item Write down the inflection point(s). 
\end{enumerate}
\end{document}


