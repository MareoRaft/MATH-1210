\documentclass[reqno,psamsfonts]{amsart}


%-------Packages---------
\usepackage[margin=.5in]{geometry}
\usepackage{nopageno}
\usepackage{amssymb,amsfonts}
\usepackage[all,arc]{xy}
\usepackage{enumerate}
\usepackage{mathrsfs}
\usepackage{tikz}
\usepackage{tikz-cd}
\usepackage{graphicx}
\usepackage{tkz-euclide}

\DeclareMathOperator{\Hom}{Hom}

%--------Theorem Environments--------
%theoremstyle{plain} --- default
\newtheorem{thm}{Theorem}[section]
\newtheorem{cor}[thm]{Corollary}
\newtheorem{prop}[thm]{Proposition}
\newtheorem{lem}[thm]{Lemma}
\newtheorem{conj}[thm]{Conjecture}
\newtheorem{quest}[thm]{Question}

\theoremstyle{definition}
\newtheorem{defn}[thm]{Definition}
\newtheorem{defns}[thm]{Definitions}
\newtheorem{con}[thm]{Construction}
\newtheorem{exmp}[thm]{Example}
\newtheorem{exmps}[thm]{Examples}
\newtheorem{notn}[thm]{Notation}
\newtheorem{notns}[thm]{Notations}
\newtheorem{addm}[thm]{Addendum}
\newtheorem{exer}[thm]{Exercise}

\theoremstyle{remark}
\newtheorem{rem}[thm]{Remark}
\newtheorem{rems}[thm]{Remarks}
\newtheorem{warn}[thm]{Warning}
\newtheorem{sch}[thm]{Scholium}

\makeatletter
\let\c@equation\c@thm
\makeatother
\numberwithin{equation}{section}

\bibliographystyle{plain}

%--------Meta Data: Fill in your info------
\title{Math 1210, section 011\\Common Mistakes With Limits}


\begin{document}
\maketitle
\thispagestyle{empty}
Compute the following limits or write DNE if the limit does not exist.
\begin{enumerate}
\item[(a)] $\lim\limits_{x\to -7}\dfrac{x^2+12x+35}{x+7}$
\\
\item[(b)] $\lim\limits_{x\to 0}\dfrac{x^2+x}{x+10}$
\end{enumerate}
\vspace{1em}
Here is how to \textbf{CORRECTLY} write the answer to these questions:
\begin{enumerate}
\item[(a)] \textbf{Solution:}
\begin{align*}
\lim\limits_{x\to -7}\frac{x^2+12x+35}{x+7}&=\lim\limits_{x\to-7}\frac{(x+5)(x+7)}{x+7}\\
&=\lim\limits_{x\to -7}x+5\\
&=-7+5\\
&=-2
\end{align*}

\item[(b)] \textbf{Solution:}
\begin{align*}
\lim\limits_{x\to 0}\frac{x^2+x}{x+10}=\frac{0^2+0}{0+10}=\frac{0}{10}=0
\end{align*}
\end{enumerate}
\vspace{2em}
Here is a list of common ways to \textbf{INCORRECTLY} answer these question:

\begin{enumerate}
\item \textbf{Scribbling scratch work on the side and writing final answer without showing any logical steps.}
\\
\item \textbf{Failure to use limit signs.}
\\
\\\underline{Question (a) example solution}:
\begin{align*}
\lim\limits_{x\to -7}\frac{x^2+12x+35}{x+7}&=\frac{(x+5)(x+7)}{x+7}\\
&=x+5\\
&=-2
\end{align*}


\item \textbf{Failure to write the expression after the limit sign.}
\\
\\\underline{Question (a) example solution}:
\begin{align*}
\lim\limits_{x\to -7}\frac{x^2+12x+35}{x+7}&=\lim\limits_{x\to-7}\frac{(x+5)(x+7)}{x+7}\\
&=\lim\limits_{x\to -7}x+5\\
\lim\limits_{x\to -7} &=-2
\end{align*}
The last equality is nonsensical. The limit sign with no expression next to it is meaningless.

\newpage
\item \textbf{Failure to use equalities.}
\\
\\\underline{Question (a) example solution}:
\begin{align*}
\lim\limits_{x\to -7}\frac{x^2+12x+35}{x+7}&\ \lim\limits_{x\to-7}\frac{(x+5)(x+7)}{x+7}\\
&\lim\limits_{x\to -7}x+5\\
&-7+5\\
&\boxed{-2}
\end{align*}

\item \textbf{Using arrows instead of equalities.}
\\
\\\underline{Question (a) example solution}:
\begin{align*}
\lim\limits_{x\to -7}\frac{x^2+12x+35}{x+7}&\implies \lim\limits_{x\to-7}\frac{(x+5)(x+7)}{x+7}\\
&\implies \lim\limits_{x\to -7}x+5\\
&\implies -7+5\\
&\implies -2
\end{align*}
In mathematics, if $P$ and $Q$ are statements, then $P\implies Q$ means $P$ implies $Q$. It does not make sense to say
\begin{align*}
\lim\limits_{x\to -7}\frac{x^2+12x+35}{x+7}&\implies \lim\limits_{x\to-7}\frac{(x+5)(x+7)}{x+7}
\end{align*}
 because the things on the left and right of the arrows are not statements, they are numbers. An example of an appropriate time to use an arrow is the following:
\begin{align*}
 x+1=6\implies x=5
\end{align*}
This time the arrow makes sense because $x+1=6$ and $x=5$ are both statements. $x+1=6$ is the statement ``$x$ plus one is equal to 6'' and $x=5$ is the statement ``$x$ is equal to 5''.
\\
\item \textbf{Misplaced equal signs.}
\\
\\\underline{Question (a) example solution}:
\begin{align*}
\lim\limits_{x\to -7}=\frac{x^2+12x+35}{x+7}&\lim\limits_{x\to-7}=\frac{(x+5)(x+7)}{x+7}\\
&\lim\limits_{x\to -7}=x+5\\
&=-7+5\\
&= -2
\end{align*}
The equal sign should always go before the limit sign, not between the limit sign and the expression.
\\
\item \textbf{Setting limit equal to $\dfrac{0}{0}$.}
\\
\\\underline{Question (a) example solution}:
\begin{align*}
\lim\limits_{x\to -7}\frac{x^2+12x+35}{x+7}=\frac{(-7)^2+12(-7)+35}{-7+7}=\frac{0}{0}
\end{align*}
The limit of a function either does not exist or is a number. It does not make sense to say that the limit of a function is equal to $\frac{0}{0}$. Remember if you plug in and get $\frac{0}{0}$ it means that more work is required. You must perform algebraic operations to reduce the expression and compute the limit or determine it does not exist.
\\
\item \textbf{Getting $\frac{0}{a}$ where $a$ is some non-zero real number and determining that the limit DNE.}
\\
\\\underline{Question (b) example solution}:
\begin{align*}
\lim\limits_{x\to 0}\frac{x^2+x}{x+10}=\frac{0^2+0}{0+10}=\frac{0}{10}=DNE
\end{align*}
If $a$ is a non-zero real number, then $\frac{0}{a}=0$.

\end{enumerate}



\end{document}


