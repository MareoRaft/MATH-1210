\documentclass[12pt]{article}
 \usepackage[margin=2cm, bottom=3cm, top=2cm]{geometry}
% \usepackage[margin=1.0in]{geometry}
% \usepackage{textpos}
% \everymath{\displaystyle}

\usepackage{amsmath,amsfonts,amsthm,amssymb}
\usepackage{extramarks}
\usepackage{fancyhdr}
\usepackage{ifthen}
\usepackage{enumitem}
\usepackage{notesheets/notesheets}

\pagestyle{fancy}

\newtheoremstyle{mainstyle}
  {20pt plus 4pt minus 4pt} % Space above
  {70pt plus 4pt minus 4pt} % Space below
  {} % Body font
  {} % Indent amount
  {\bfseries} % Theorem head font
  {.} % Punctuation after theorem head
  {.5em} % Space after theorem head
  {} % Theorem head spec (can be left empty, meaning `normal')
\newtheoremstyle{longstyle}
  {20pt plus 4pt minus 4pt} % Space above
  {170pt plus 4pt minus 4pt} % Space below
  {} % Body font
  {} % Indent amount
  {\bfseries} % Theorem head font
  {.} % Punctuation after theorem head
  {.5em} % Space after theorem head
  {} % Theorem head spec (can be left empty, meaning `normal')

\theoremstyle{mainstyle} % this applies to ALL following new theorems % SEE research seminar, week 10 writeup for a possible way to exclude amsthm and [thm].


\newtheorem{thrm}[thm]{Theorem}
\newtheorem{defi}[thm]{Definition}
\newtheorem{coro}[thm]{Corollary}
\newtheorem{propo}[thm]{Proposition}
\newtheorem{lemm}[thm]{Lemma}
\newtheorem{examp}[thm]{Example}

\theoremstyle{longstyle}
\newtheorem{ex}[thm]{Challenge}



\everymath{\displaystyle}

\newcommand{\vs}{\vspace{8pt}}


\pagenumbering{gobble}
\setlength\parindent{0pt}
%
% Create Problem Sections
%
\newcommand{\enterProblemHeader}[1]{
\nobreak\extramarks{}{Problem \arabic{#1} continued on next page\ldots}\nobreak\
{}
\nobreak\extramarks{Problem \arabic{#1} (continued)}{Problem \arabic{#1} contin\
ued on next page\ldots}\nobreak{}
}
\newcommand{\exitProblemHeader}[1]{
\nobreak\extramarks{Problem \arabic{#1} (continued)}{Problem \arabic{#1} contin\
ued on next page\ldots}\nobreak{}
\stepcounter{#1}
\nobreak\extramarks{Problem \arabic{#1}}{}\nobreak{}
}

\setcounter{secnumdepth}{0}
\newcounter{partCounter}
\newcounter{homeworkProblemCounter}
\setcounter{homeworkProblemCounter}{1}
\nobreak\extramarks{Problem \arabic{homeworkProblemCounter}}{}\nobreak{}
%
% Homework Problem Environment
%
% This environment takes an optional argument. When given, it will
% adjust the
% problem counter. This is useful for when the problems given for your
% assignment aren't sequential. See the last 3 problems of this
% template for an
% example.
%                                                          ~%
%
% Homework Problem Environment
%
% This environment takes an optional argument. When given, it will adjust the
% problem counter. This is useful for when the problems given for your
% assignment aren't sequential. See the last 3 problems of this template for an
% example.
%

\newenvironment{homeworkProblem}[1][-1]{
\ifnum#1>0
\setcounter{homeworkProblemCounter}{#1}
\fi
\subsubsection{Problem \arabic{homeworkProblemCounter}}
\setcounter{partCounter}{1}
\enterProblemHeader{homeworkProblemCounter}
}{
  \vspace{35ex}
  \exitProblemHeader{homeworkProblemCounter}
}
\newcommand{\footer}{
\rfoot{\ifthenelse{\value{page}=1}{\textbf{This quiz has 2 sides}}{}}
}
\newcommand{\header}{
a\lhead{\large MATH 1210}
\chead{\large Implicit Differentiation \& 1st Deriv
  Applications Quiz} 
\rhead{\large Fall 2017}
{{Name and computing ID:} \underline{\hspace{3in}}}\\ \ \\
\fbox{
\begin{minipage}{6.5 in}
\textit{On my honor as a student, I pledge that I have neither given nor received help on this assignment.} \\ \ \\
{Signature:} {\underline {\hspace{3in}}}
\end{minipage}}
}

\everymath{\displaystyle}

%%%%%%%%%%%%%%%%%%%%%%%%%%%%%%%%%%%%%%%%%%%%%
\author{Math 1210}
\title{Curve Sketching Practice Problems}

\begin{document}
\maketitle
% \begin{textblock*}{100mm}(.65\textwidth,-5.3cm)
%   Name:\underline{\qquad\qquad\qquad\qquad\qquad\qquad}
% \end{textblock*}
%%%%%%%%%%%%%%%%%%%%%%%%%%%%%%%%%%%%%%%%%%%%%
\begin{homeworkProblem}
  Let \[
    f(x) = 2x^3+3x^2-12x
  \]
  Sketch the graph of $f$. Clearly label any $x$ and $y$ intercepts, horizontal asymptotes, vertical asymptotes, relative max pairs, relative min pairs, and inflection points. 
\end{homeworkProblem}
\begin{homeworkProblem}
  Suppose $f$ is a function with all of the following properties:
\begin{enumerate}
\item $f$ has domain$=(-\infty, 2)\cup (2,\infty)$.
\item $f$ has x-intercepts $-3$, $4$, $7$, and $9$. $f$ has y-intercept $6$. 
\item $\lim\limits_{x\to\infty}f(x) = \infty$ and $\lim\limits_{x\to-\infty}f(x) = -5$. Also, $f$ has a vertical asymptote $x=2$. $f$ has the following behavior near this vertical asymptote:
\begin{align*}
\lim\limits_{x\to 2^{+}}f(x) = -\infty\hspace{2em}\text{and}\hspace{2em}\lim\limits_{x\to 2^{-}}f(x) = \infty
\end{align*}
\item $f$ is increasing on $(-\infty, 2)$, $(2, 5)$, and $(8, \infty)$. $f$ is decreasing on $(5,8)$
\item $f$ has a relative max pair of $(5,3)$ and a relative min pair of $(8, -1)$. 
\item $f$ is concave up on $(-\infty, 2)$ and $(6, \infty)$. $f$ is concave down on $(2, 6)$. 
\item $f$ has an inflection point $(6, 1)$. 
\\
\end{enumerate}
  Sketch a possible graph of $f$. In other words, sketch the graph of a single function which has all of the properties listed above. Clearly label any $x$ and $y$ intercepts, horizontal asymptotes, vertical asymptotes, relative max pairs, relative min pairs, and inflection points.
\end{homeworkProblem}
\begin{homeworkProblem}
  Let \[
f(x) = \dfrac{x^2-8x-9}{x(x+1)}
  \]
  Sketch the graph of $f$. Clearly label any $x$ and $y$ intercepts, horizontal asymptotes, vertical asymptotes, relative max pairs, relative min pairs, and inflection points.
\end{homeworkProblem}
\end{document}