\documentclass[12pt, a4paper]{article}
\usepackage{../notesheets}
\usepackage{epsdice}

\newcommand{\Var}{\operatorname{Var}}
%%%%%%%%%%%%%%%%%%%%%%%%%%%%%%%%%%%%%%%%%%%%%%%%%%
\author{Math 1220}
\title{Notesheet. Section 11.3: Series} 
\date{}

\begin{document}
\maketitle
\nameline
%%%%%%%%%%%%%%%%%%%%%%%%%%%%%%%%%%%%%%%%%%%%%%%%%%
\begin{defi}
  A \de{series} is
\end{defi}
\begin{defi}
  Given a sequence \(\{a_n\}_{n=k}^\infty\), the \de{\(N\)th partial
    sum} for \(N \geq k\) is \\
\end{defi}
\begin{defi}
  We say the series \(\sum_{n=k}^\infty a_n\) is \de{convergent} if \\

  We say \(\sum_{n=k}^\infty a_n\) is divergent if 
\end{defi}
\begin{ex}
  Is \(\sum_{n=2}^\infty \frac{3}{2^n}\) convergent or divergent?
\end{ex}
\begin{defi}
  The series above is a geometric series. A \de{geometric series} with
  ratio \(r\) is a series of the form
\end{defi}
\begin{thrm}
  A geometric series \(\sum_{n=0}^\infty ar^n\) is convergent with sum
  \\

  and it is divergent if 
\end{thrm}
\vspace{-1in}
\begin{ex}
  Express the decimal \(0.131313\cdots\) as a fraction of
  integers. (Hint: Write \(0.1313\cdots\) as a geometric series.)
\end{ex}
\begin{thrm}
  If \(\sum_{n=1}^\infty a_n\) and \(\sum_{n=1}^\infty b_n\) are
  convergent infinite series and \(c\) is a constant, then
  \begin{enumerate}
  \item \(\sum_{n=1}^\infty ca_n = \)
  \item \(\sum_{n=1}^\infty (a_n \pm b_n) = \)
  \end{enumerate}
\end{thrm}
\vspace{-1in}
\begin{defi}
  A \de{telescoping series} is a series \(\sum_{n=k}^\infty a_n\) such that
\end{defi}
\begin{ex}
  Determine the convergence of the following (telescoping) series using
  the partial 
  sum definition.
  \begin{enumerate}
  \item \(\sum_{n=1}^\infty \left( \frac{1}{n} - \frac{1}{n+1}
    \right)\)
    \vspace{1in}
  \item \(\sum_{n=1}^\infty \frac{1}{n^2+n}\)
  \end{enumerate}
\end{ex}
\vspace{-1.5in}
%%%%%%%%%%%%%%%%%%%%%%%%%%%%%%%%%%%%%%%%%%%%%%%%%% 
\end{document}
