\documentclass[12pt, a4paper]{article}
\usepackage{../notesheets}
%%%%%%%%%%%%%%%%%%%%%%%%%%%%%%%%%%%%%%%%%%%%%%%%%%
\author{Math 1210}
\title{Notesheet. Section 12.3: Differentiation of Trigonometric Functions}
\date{}

\begin{document}
\maketitle
\nameline
%%%%%%%%%%%%%%%%%%%%%%%%%%%%%%%%%%%%%%%%%%%%%%%%%%
\begin{thrm}
  We have the following derivatives.
  \begin{enumerate}
  \item \(\frac{d}{dx} \sin(x) = \)
  \item \(\frac{d}{dx} \cos(x) = \)
  \end{enumerate}
\end{thrm}
\begin{ex}
  Using the quotient rule, what is \(\frac{d}{dx} \tan(x)\)?
\end{ex}
\begin{ex}
  Using the chain rule, what is \(\frac{d}{dx} \sec(x)\)? 
\end{ex}
\begin{thrm}[Trigonometric Derivatives]
  We have the following derivatives.\\
  \begin{minipage}{0.5\linewidth}
    \begin{itemize}
    \item \(\frac{d}{dx} \sin(x) = \)
    \item \(\frac{d}{dx} \sec(x) = \)
    \item \(\frac{d}{dx} \tan(x) = \)
    \end{itemize}
  \end{minipage}
  \begin{minipage}{0.4\linewidth}
    \begin{itemize}
    \item \(\frac{d}{dx} \cos(x) = \)
    \item \(\frac{d}{dx} \csc(x) = \)
    \item \(\frac{d}{dx} \cot(x) = \)
    \end{itemize}
  \end{minipage}
  \ \\
  
  Useful mnemonic: starts with `c' \(\iff\) the derivative has a minus
  sign.
\end{thrm}
\begin{ex}
  Find an equation of the tangent line of \(f(x) = \sin(x^2)\) at
  \(\left( \sqrt{\frac{\pi}{2}}, 1 \right)\). 
\end{ex}
\begin{ex}
  A rocket is blasting off into space by launching
  vertically. Let \(y(t)\) be a function indicating how high the
  rocket is at time \(t\). \(12,000\) feet away, a camera is setup to
  watch the 
  rocket. Let \(\theta\) be the angle between the line from the camera
  to the launch point and the line between the camera and the rocket.
  How fast is \(\theta\) changing at the instant when  
  \begin{enumerate}
  \item the rocket is at a distance of \(13,000\) feet from the camera
    and
  \item the distance is increasing at the rate of \(480\) feet/second?
  \end{enumerate}

\end{ex}
%%%%%%%%%%%%%%%%%%%%%%%%%%%%%%%%%%%%%%%%%%%%%%%%%% 
\end{document}
