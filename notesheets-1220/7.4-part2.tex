\documentclass[12pt, a4paper]{article}
\usepackage{../notesheets}
%%%%%%%%%%%%%%%%%%%%%%%%%%%%%%%%%%%%%%%%%%%%%%%%%%
\author{Math 1220}
\title{Notesheet. Section 7.4: Improper Integrals Part II}
\date{}

\begin{document}
\maketitle
\nameline
%%%%%%%%%%%%%%%%%%%%%%%%%%%%%%%%%%%%%%%%%%%%%%%%%%
\begin{defi}
  If there is a real number \(c\) such that \emph{both} of
  \(\int_{-\infty}^c f(x) \dx\) and \(\int_c^{\infty} f(x) \dx\) are
  convergent, then we say \(\int_{-\infty}^\infty f(x) \dx\) is \\

  Furthermore, we compute it as \[
    \int_{-\infty}^\infty f(x) \dx = \hspace{3in}
  \]
  Otherwise, if \(\int_{-\infty}^c f(x)\dx\) or \(\int_c^\infty f(x)
  \dx\) are divergent for some \(c\), then we say
  
  \(\int_{-\infty}^\infty f(x) \dx\) is
\end{defi}
\begin{ex}
  Evaluate \(\int_{-\infty}^\infty \frac{2x}{x^2+1} \dx\) if it
  converges. Also try \(\int_{-\infty}^\infty \frac{2x}{(x^2+1)^2} \dx\).
\end{ex}
\begin{ex}
  Determine if \(\int_{-\infty}^\infty x \dx\) converges. Before using
  any calculus, what does your intuition tell you?
\end{ex}
\begin{ex}
  True or False? \(\int_{-\infty}^\infty f(x) \dx = \lim_{t \to
    \infty} \int_{-t}^t f(x)\dx\)?
\end{ex}
\begin{ex}
  Evaluate \(\int_{-\infty}^\infty x e^{-\frac{1}{2}x^2} \dx\) if it
  converges. 
\end{ex}
\begin{ex}
  Using some of what we have already seen, for which real numbers
  \(p\) is the following integral convergent? \[
    \int_1^\infty \frac{1}{x^p} \dx
  \]
\end{ex}
\begin{rmk}
  Later we will use the fact that \[
    \int_{-\infty}^\infty e^{-\frac{1}{2}x^2} \dx = 
  \]
  for many probability problems.
\end{rmk}
%%%%%%%%%%%%%%%%%%%%%%%%%%%%%%%%%%%%%%%%%%%%%%%%%% 
\end{document}
