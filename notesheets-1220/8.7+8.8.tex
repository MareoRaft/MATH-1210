\documentclass[12pt, a4paper]{article}
\usepackage{../notesheets}
%%%%%%%%%%%%%%%%%%%%%%%%%%%%%%%%%%%%%%%%%%%%%%%%%% 
\author{Math 1220}
\title{Notesheet. Section 8.7+8.8: Double Integrals + Geometric Applications}
\date{}

\begin{document}
\maketitle
\nameline
%%%%%%%%%%%%%%%%%%%%%%%%%%%%%%%%%%%%%%%%%%%%%%%%%%
\begin{defi}
  The \de{double integral} of \(f(x,y)\) over a region \(R\) is
  denoted by \[
    \iint_R f(x,y)\ dA = 
  \]
  and, if \(f(x,y) \geq 0\) over \(R\), then \[
    \iint_R f(x,y)\ dA = 
  \]
\end{defi}
\begin{rmk}
  \begin{enumerate}
  \item In practice, double integrals are evaluated using suitable iterated
  integrals. 
  \item A consequence of our definition is that, for a region \(R\),
    \(\iint_R 1\ dA\) numerically gives
  \end{enumerate}
\end{rmk}
\begin{thrm}
  If \(R\) is a rectangular region defined by \(a \leq x \leq b\) and
  \(c \leq y \leq d\), then \[
    \iint_R f(x,y)\ dA = 
  \]
\end{thrm}
\begin{ex}
  Evaluate the double integral \(\iint_R 2xy \dx\ dy\) where \(R\) is
  the region defined by the inequalities \(0 \leq x \leq 1\) and \(0
  \leq 1 \leq y\).
\end{ex}
\begin{thrm}
  \begin{enumerate}
  \item   Suppose \(g_1(x)\) and \(g_2(x)\) are continuous functions on
  \([a,b]\) and the region \(R\) is defined by \(R = \{(x,y) \mid
  g_1(x) \leq y \leq g_2(x); a \leq x \leq b\}\). Then, \[
    \iint_R f(x,y)\ dA = 
  \]
  \item Suppose \(h_1(y)\) and \(h_2(y)\) are continuous functions on
    \([c,d]\) and the region \(R\) is defined by \(R = \{(x,y) \mid
    h_1(y) \leq x \leq h_2(y); c \leq y \leq d\}\). Then, \[
      \iint_R f(x,y)\ dA = 
    \]
  \end{enumerate}
\end{thrm}
\begin{ex}
  Let \(R\) be the region bounded by \(x=-y\) and \(x=y\) for \(0 \leq
  y \leq 1\). Evaluate \[
    \iint_R 2xy\ dA
  \]
\end{ex}
\begin{ex}
  Let \(R\) be the region bounded by \(y=x^2\) and
  \(y=4\). Evaluate \[
    \iint_R y\ dA
  \]
\end{ex}
%%%%%%%%%%%%%%%%%%%%%%%%%%%%%%%%%%%%%%%%%%%%%%%%%% 
\end{document}