\documentclass[12pt, a4paper]{article}
\usepackage{../notesheets}
\usepackage{epsdice}
\usepackage{tikz}
\usetikzlibrary{shapes.geometric, arrows}

\newcommand{\Var}{\operatorname{Var}}
%%%%%%%%%%%%%%%%%%%%%%%%%%%%%%%%%%%%%%%%%%%%%%%%%%
\author{Math 1220}
\title{Notesheet. Sections 11.1+11.6: Taylor Polynomials and More
  Taylor Series} 
\date{}

\begin{document}
\maketitle
\nameline
%%%%%%%%%%%%%%%%%%%%%%%%%%%%%%%%%%%%%%%%%%%%%%%%%%
\begin{defi}
  The \(N\)th Taylor polynomial \(P_N(x)\) of \(f(x)\) at \(x=a\) is 
\end{defi}
\begin{ex}
  Find the 2nd Taylor polynomial of \(f(x) = e^x\) at \(x=0\) and use it to
  approximate the decimal value of \(e\).
\end{ex}
\begin{ex}
  Find \(P_2(x)\) for \(f(x) = e^{-\frac{1}{2}x^2}\) at \(x=0\). Use
  \(P_2(x)\) to approximate \(P(0 < Z < 1)\) for standard normal RV \(Z\).
\end{ex}
\pagebreak
\begin{thrm}
  If \(f(x) = \sum a_n x^n\) on interval of convergence \((-R < x <
  R)\), then \[
    f(u(x)) = \hspace{1in}
  \]
\end{thrm}
\vspace{-0.75in}
\begin{ex}
  Find the Maclaurin series of the following functions and their
  intervals of convergence
  \begin{enumerate}
  \item \(f(x) = \frac{1}{1-2x}\)
    \vspace{1.5in}
  \item \(f(x) = e^{x^5}\)
  \end{enumerate}
\end{ex}
\vspace{-0.5in}
\begin{thrm}
  If \(f(x) = \sum a_n x^n\) on interval of convergence \(I\), then \[
    x^p f(x) = \hspace{1in}
  \]
\end{thrm}
\vspace{-0.5in}
\begin{ex}
  Find the Maclaurin series of the following functions and their
  intervals of convergence
  \begin{enumerate}
  \item \(f(x) = \frac{x^3}{1-2x}\)
    \vspace{1in}
  \item \(f(x) = \frac{\ln(x+1)}{x}\)
  \end{enumerate}

\end{ex}
%%%%%%%%%%%%%%%%%%%%%%%%%%%%%%%%%%%%%%%%%%%%%%%%%% 
\end{document}
