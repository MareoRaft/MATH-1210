\documentclass[12pt, a4paper]{article}
\usepackage{../notesheets}
%%%%%%%%%%%%%%%%%%%%%%%%%%%%%%%%%%%%%%%%%%%%%%%%%% 
\author{Math 1220}
\title{Notesheet. Section 8.4+8.7: The Method of Least Squares +
  Double Integrals}
\date{}

\begin{document}
\maketitle
\nameline
%%%%%%%%%%%%%%%%%%%%%%%%%%%%%%%%%%%%%%%%%%%%%%%%%%
\begin{defi}
  Given \(n\) data points \((x_1,y_1), \ldots, (x_n,y_n)\), a \de{scatter
    diagram} is 
\end{defi}
\begin{defi}
  The \de{principle of least squares} states that the straight line
  \(L\) that fits \(n\) data points best is 
  \\\vspace{0.2in}\\
  The line \(L\) obtained in this manner is called the
  \de{least-squares line}, or \de{regression line}.
\end{defi}
\vspace{-1in}
\begin{thrm}
  Given \(n\) data points \((x_1,y_1), (x_2,y_2), \ldots, (x_n,y_n)\),
  then the least squares (regression) line for the data is given by
  the linear equation \[
    y = f(x) = mx + b
  \]
  where the constants \(m\) and \(b\) satisfy the equations \[
    \begin{cases}
      \hspace{3in}\vspace{0.5in} \\
      \
    \end{cases}
  \]
\end{thrm}
You will see why this is true in homework.
\begin{ex}
  The following data consists of the quiz grades for five students \\
    \begin{tabular}{|c|c|c|}
      \hline \text{Student}&\text{Quiz 1 Grade}&\text{Quiz 2 grade} \\
      \hline 1&\(x_1 = 1\)& \(y_1 = 1\)\\
      2&\(x_2 = 2\)&\(y_2 = 3\)\\
      3&\(x_3=3\)&\(y_3 = 4\)\\
      4&\(x_4=4\)&\(y_4=3\)\\
      5&\(x_5=5\)&\(y_5=6\)\\
      \hline
    \end{tabular}
  \end{ex}
  \begin{ex}
    Evaluate the following integrals by integrating with respect to
    the appropriate variable and treating the other variable as a constant.
    \begin{enumerate}
    \item \(\int_0^1 xy^2 \dx\)
      \vspace{1in}
    \item \(\int_0^x (x+y)\ dy\)
      \vspace{1in}
    \item \(\int_1^2 \frac{x}{y^2} e^{x/y}\ dy\)
      \vspace{1in}
    \end{enumerate}
  \end{ex}
  \vspace{-2in}
  \begin{ex}
    Now, evaluate the \de{interated integrals} by doing the inside
    integral and then the outside integral.
    \begin{enumerate}
    \item \(\int_0^1 \int_0^1 x y^2 \dx\ dy\)
      \vspace{1in}
    \item \(\int_0^1 \int_0^x (x+y)\ dy \dx\)
      \vspace{1in}
    \item \(\int_0^x \int_0^1 (x+y) \dx\ dy\)
    \end{enumerate}

  \end{ex}
%%%%%%%%%%%%%%%%%%%%%%%%%%%%%%%%%%%%%%%%%%%%%%%%%% 
\end{document}