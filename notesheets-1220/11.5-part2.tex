\documentclass[12pt, a4paper]{article}
\usepackage{../notesheets}
\usepackage{epsdice}
\usepackage{tikz}
\usetikzlibrary{shapes.geometric, arrows}

\newcommand{\Var}{\operatorname{Var}}
%%%%%%%%%%%%%%%%%%%%%%%%%%%%%%%%%%%%%%%%%%%%%%%%%%
\author{Math 1220}
\title{Notesheet. Section 11.5 Part 2: Power Series and Taylor Series} 
\date{}

\begin{document}
\maketitle
\nameline
%%%%%%%%%%%%%%%%%%%%%%%%%%%%%%%%%%%%%%%%%%%%%%%%%%
\begin{defn}
  The \de{Taylor series} of \(f(x)\) at \(x=a\) is a power series 
\end{defn}
\vspace{0.25in}
\begin{ex}
  Find the Taylor series of
  \begin{enumerate}
  \item \(f(x) = \frac{1}{x-1}\) at \(x=2\)
    \vspace{2in}
  \item \(f(x) = \ln(1+x)\) at \(x=0\)
  \end{enumerate}
\end{ex}
\begin{defn}
  The \de{Maclaurin series} of \(f(x)\) is 
\end{defn}
\vspace{0.25in}
\pagebreak
\begin{ex}
  Find the Maclaurin series of \(f(x) = xe^x\).
\end{ex}
\begin{ex}
  The following are common Maclaurin series which you should
  know. Convince yourself of them at home!
  \begin{enumerate}
  \item \(e^x = \sum_{n=0}^\infty\)
  \item \(\ln(1+x) = \sum_{n=1}^\infty\)
  \item \(\sin(x) = \sum_{n=0}^\infty\)
  \item \(\cos(x) = \sum_{n=0}^\infty\)
  \item \(\frac{1}{1-x} = \sum_{n=0}^\infty\)
  \end{enumerate}
\end{ex}
\vspace{-2in}
\begin{thrm}
  If \(\sum_{n=0}^\infty a_n(x-a)^n\) represents \(f(x)\) at \(x=a\),
  then \[
    \frac{d}{dx} \sum_{n=0} a_n(x-a)^n = \hspace{5in}
  \]
\end{thrm}
\begin{thrm}
  If \(\sum_{n=0}^\infty a_n (x-a)^n\) represents \(f(x)\) at \(x=a\),
  then \(a_n = \)
\end{thrm}
%%%%%%%%%%%%%%%%%%%%%%%%%%%%%%%%%%%%%%%%%%%%%%%%%% 
\end{document}
