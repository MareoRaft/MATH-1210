\documentclass[12pt, a4paper]{article}
\usepackage{../notesheets}
\usepackage{epsdice}

\newcommand{\Var}{\operatorname{Var}}
%%%%%%%%%%%%%%%%%%%%%%%%%%%%%%%%%%%%%%%%%%%%%%%%%%
\author{Math 1220}
\title{Notesheet. Section 10.1+10.2: Joint Distributions, Expected
  Value, and Standard Deviation} 
\date{}

\begin{document}
\maketitle
\nameline
%%%%%%%%%%%%%%%%%%%%%%%%%%%%%%%%%%%%%%%%%%%%%%%%%%
\vspace{-0.3in}
\begin{defi}
  A \de{joint probability dnesity function} of the random variables
  \(X\) and \(Y\) on a region \(D\) is a function \(f(x,y)\) such that
\end{defi}
\begin{thrm}
  The event \(E = ``(X,Y)''\) in \(R\) has probability \[
    P[(X,Y) \text{ in } R] = \hspace{1in}
  \]
\end{thrm}
\vspace{-0.5in}
\begin{ex}
  Show that \[
    f(x,y) = \frac{1}{4}(x+2y); D = \{0 \leq x \leq 2; 0 \leq y \leq 1\}
  \]
  is a joint probability density function. Find the probability \(P(0
  \leq X \leq 1; 0.5 \leq Y \leq 1)\).
\end{ex}
\begin{defi}
  The \de{expected value} or \de{mean} \(E(X)\) of a random variable
  \(X\) on an interval \(I\) is given by \[
    E(X) = \hspace{1in}
  \]
  and refers to ``the value one would expect to find if one repeated
  the experiment infinitely many times.''
\end{defi}
\begin{ex}
  Let \(f(x) = \frac{e^x}{e-1}\)  be the PDF for \(X\) on
  \(I=[0,1]\). Find \(E(X)\).
\end{ex}
\begin{defi}
  The \de{variance} of a random variable \(X\) on \(I\) is given by \[
    \Var(X) = \hspace{1in}
  \]
  and the \de{standard deviation} of \(X\) is \[
    \sigma = \hspace{1in}
  \]
\end{defi}
\vspace{-0.75in}
\begin{rmk}
  If we rewrite \(E(X) = \mu\), since \(E(X)\) is a constant, then we
  get an easier formula for variance given by \[
    \Var(X) = \hspace{1in}
  \]
\end{rmk}
\begin{ex}
  Let \(f(x) = \frac{e^x}{e-1}\)  be the PDF for RV \(X\) on
  \(I=[0,1]\). Find \(\Var(X)\).
\end{ex}
\begin{ex}
  Assume \(X\) is a \(RV\) on \(I\) with \(\Var(X) = 2\) and \(\int_I
  x^2 f(x) \dx = 11\). Find \(E(X)\).
\end{ex}
%%%%%%%%%%%%%%%%%%%%%%%%%%%%%%%%%%%%%%%%%%%%%%%%%% 
\end{document}
