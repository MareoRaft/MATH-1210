\documentclass[11pt]{article}
\textwidth=7in
\textheight=9.5in
\topmargin=-1in
\headheight=0in
\headsep=.5in
\hoffset -.85in

\usepackage{enumerate,hyperref,multicol,color}
\newcommand{\red}[1]{\textcolor{red}{#1}}
\newcommand{\blue}[1]{\textcolor{blue}{#1}}
\newcommand{\webkey}{\red{virginia 9636 9041}}
\newcommand{\vs}{\vspace{0.5cm}}


\pagestyle{empty}
\begin{document}

\begin{center}
{\bf University of Virginia \\ Course Syllabus: MATH 1210--Section 2 \\ A Survey of Calculus I \\ Spring 2018}
\end{center}


\begin{flushleft}
\rule{\textwidth}{.01in}
\end{flushleft}

\begin{multicols}{2}
\noindent\textbf{Instructor:} Matthew Lancellotti \\
\noindent\textbf{Office:} Kerchof 110 \\
\noindent\textbf{contact:} math1210section2.slack.com \\
\noindent\textbf{Class location:} New Cabell Hall 323\\
\noindent\textbf{Class meeting time:} MoWe 2:00PM - 3:15PM \\
\noindent\textbf{course website:} \href{https://math1210section2.slack.com}{\blue{math1210section2.slack.com}}
\end{multicols}
\noindent\textbf{Office Hours:} MoWeTh 10--11:00 a.m. (subject to change)

\vs\vs
\noindent\textbf{Course communication/website:}
The main web portal from which you can access EVERYTHING about this course (and communicate with me as well as each other) is Slack (The link is above, but the FIRST time you enter, you must use this \href{https://join.slack.com/t/math-1210-section-2/shared_invite/enQtMjk5NjMxNzUzNjUyLWVhMzYxZGU5MzcyMGE1NWUyMzRmNWU4ODU1N2RkMTRkMjlhM2VmODEzYTMxYzgwMzExNzY5NGUxMDBjMTc1NzM}{INVITE LINK} to gain access: \url{http://bit.ly/2rc13nx}).  If for some reason you are not already signed up, talk to me after class.



\vs
\noindent\textbf{Course Description:}
Math 1210 is an introductory calculus course intended for students interested primarily in the life, managerial, and social sciences. Math 1210 is a coordinated course. This means that all sections cover the same material and take the same tests.

Calculus might be defined as a mathematical toolkit for analyzing functions. In virtually every area of human endeavor, functions are or can be used to further understanding and to assist in making predictions.

\begin{itemize}
\item A biologist might be interested in population as a function of time.
\item A medical researcher might be interested in modeling blood pressure as a function of body weight, or concentration of a drug in the bloodstream as a function of time since ingestion.
\item A business executive might study the demand for a product as a function of its price, or, perhaps, as a function of size of marketing budget.
\item An environmental scientist might be interested in the level of a toxin in a lake as a function of time.
\item An astronomer might be interested in star luminosity as a function of mass.
\end{itemize}

Calculus provides \textbf{two} fundamental tools for analyzing functions: the \textit{derivative}, which represents the rate of change of a function, and the definite \textbf{integral}, which can be used to compute the net change of a function over an interval.


\vs
\noindent\textbf{Course Objectives:}
Upon successful completion of this course, students will
\begin{itemize}
\item be able to work confidently with functions represented verbally, numerically (by a table of values), graphically, or algebraically (by a formula) and be able to relate, as well as create, such representations;
\item understand, be able to describe, and be able to apply the fundamental tools that calculus provides for analyzing functions: derivatives, which represent rates of change, and definite integrals, which can be used to compute net change;
\item recognize when the tools of calculus can be applied to analyze a function and be able to communicate---with clarity and precision---the results of their analysis;
\item  by modeling and solving a variety of problems including some with real-world applications, students will develop problem-solving skills and strategies such as breaking complex problems into simpler subproblems and testing solutions for plausibility. They will also come to understand how theoretical results and concepts can be developed and then used as tools for problem solving as well as further investigation;
\item be able to assess the quality of competing solutions to problems based on criteria such as clarity, efficiency, and elegance.
\end{itemize}


\vs
\noindent\textbf{Am I in the right calculus class?} Read the Math Department placement information \\
at \href{http://www.math.virginia.edu/content/math-placement}{\blue{math.virginia.edu/content/math-placement}}.

\vs
\noindent\textbf{Textbook}: This course will cover chapters 1-6 (omitting some sections) of the course text {\it Applied Calculus for the Managerial, Life, and Social Sciences} by Soo T. Tan, 9th edition (Publisher: Brooks/Cole Cengage Learning). An electronic edition of the text is provided through the on-line homework system WebAssign, to which you must purchase access. Acquisition of a physical copy of the text is optional.  You have a number of different purchase options:
\begin{enumerate}[(1)]
\item  purchase WebAssign single-term access on-line through the WebAssign Website,
\item  purchase a single-term WebAssign-access card at the UVA Bookstore,
\item purchase a physical copy of the text, bundled with a multi-term WebAssign-access card, at the UVA Bookstore, or
\item purchase WebAssign via (1) or (2) and, if you want a hard-copy of the text, buy a used copy from the Bookstore.
\end{enumerate}

\noindent{\it There is a two-week ``grace period'' at the beginning of the term during which you have free WebAssign access to the text and course homework sets---go to \href{http://www.webassign.net/uva/login.html}{\blue{webassign.net/uva/login.html}}, and via the gray button on the upper right, enter our class key: ``\webkey''.}

\begin{center}
{\bf Assessments}
\end{center}

\vs
\noindent\textbf{Diagnostic Quiz}:
On \red{Monday, January 29}, there will be a quiz (15--20 minutes) consisting of problems designed to test your ``readiness for calculus'' skills.  Most of these problems will be similar to homework problems appearing on the first two or three WebAssign homework assignments.


\vs
\noindent\textbf{Homework}:
Most homework for this course will be delivered through the WebAssign system: go to \href{http://www.webassign.net/uva/login.html}{\blue{webassign.net/uva/login.html}} and enter our class key ``\webkey''. The system will give you immediate feedback and you will be allowed to attempt problems multiple times. You should record your work on a given problem by hand (just as if you were working through a test problem) and then enter your response into WebAssign. Keep in mind that when you respond to problems on exams and quizzes your work, as well as your answers, will be evaluated. When you have trouble with a homework problem, be alert to what you learn as you work toward a solution.



\vs
\noindent\textbf{Exams}:  Recall that one objective of the course is improvement of your problem-solving skills.  To motivate you to develop these skills as well as to give you an opportunity to show you understand how to choose and apply appropriate calculus tools in your problem solving, exams will include some problems that are somewhat different from those you've solved before (but for which you have learned tools and strategies that will produce solutions).

 There will be two evening midterm exams given during the semester. The exams are common to all sections of MATH 1210. The dates of these exams are as follows:
\vskip.05in
\noindent\textbf{Midterm Exam 1}: Thursday, February 22, 7--8:30 p.m.\\
\noindent\textbf{Midterm Exam 2}: Thursday, April 12th, 7--8:30 p.m.
\vskip.05in

\noindent For those students who have a time conflict with another course, a make-up exam will be given the following morning beginning at 7:20 a.m. If you have a direct conflict with either of the above listed exam times, please notify me as soon as possible and at least one week before the exam date. If proper notice cannot be given, then a request for the make-up exam will be honored only in cases of extreme emergencies and at my discretion.  Midterm and final exams will be graded in common, with all Math 1210 instructors participating.

\vskip.1in
The {\bf  final exam} will be given Tuesday, May 8th, 7--10:00 p.m.  This is the time reserved for the MATH 1210 final exam by the University and all sections of MATH 1210 take the common final examination at the same time. It is University policy that final exams may not be taken early. The final exam is comprehensive.


\vs
\noindent\textbf{Course Grade}:
The course grade will be determined as follows:
\begin{multicols}{2}
\noindent WebAssign homework \\
Written homework and Quizzes: \\
Midterm Exam 1: \\
Midterm Exam 2: \\
Final Examination: \\ \\
10  points   \\
15 points \\
20 points \\
25 points \\
\underline{30 points}\\
100 points possible
\end{multicols}

\noindent The number of points you earn will be mapped to a letter grade as follows:
\begin{center}
\begin{tabular}{|c||c||c||c||c||c|}
\hline
 A+: [98, 100] & A:  [93, 98) & A-: [90, 93) & B+:  [87, 90) &  B:   [83, 87) & B-:  [80. 83) \\
 \hline
 C+:   [77, 80) & C:   [73, 77) & C-:  [70, 73) & D+:  [67, 70) & D:  [63, 67)& D-: [60, 63)\\
\hline
\end{tabular}
\end{center}
In borderline cases, your letter grade may be higher---the one assigned to the interval immediately above the one your point total lies in.

\begin{center}
{\bf Policies}
\end{center}


\noindent\textbf{Attendance and Classroom Etiquette}:

Attendance is optional, with the exception of the Diagnostic Quiz, two Midterms, and Final.  (Attendance is still recommended.  Students who attend learn much more on average than students who don`t, but in rare cases students may learn better by self studying.  Also, students not attending class are still responsible for homework assignments that are partially classwork.)  Students are not permitted to use technology in class because it is distracting for other students.  However, a student who wishes to use technology for learning purposes may obtain permission by talking to me in office hours.


\vskip.1in
\noindent\textbf{Calculators}: Calculators will not be allowed for any quizzes or exams. Thus, as much as possible, try to complete homework problems without using a calculator. (For some homework problems, you will find a calculator or \href{http://www.wolframalpha.com/}{\blue{Wolfram Alpha}} to be helpful.)

\vskip.1in
\noindent\textbf{Learning Needs}:  The University of Virginia strives to provide accessibility to all students. If you require an accommodation to fully access this course, please contact the Student Disability Access Center (SDAC) at (434) 243-5180 or sdac@virginia.edu. If you are unsure if you require an accommodation, or to learn more about their services, you may contact the SDAC at the number above or by visiting their website at \url{https://sdac.studenthealth.virginia.edu}.
\smallskip
Accommodations for test-taking (e.g., extended time) should be arranged at least 5 business days before an exam.

\vskip.1in
\noindent\textbf{Honor Code}: The Honor Code will be strictly observed in this class.\footnote{Recent honor violations committed by calculus students include: falsifying a doctor's note in order to postpone a scheduled exam; presenting a false excuse for postponing an exam; and, seeking to boost an exam score by correcting mistakes on a graded, returned exam and then reporting ``grading errors'' on the exam. Note that calculus instructors scan graded exams.} Please remember to pledge each quiz and exam.
\begin{center}


\textbf{Tips for success}
\end{center}


\noindent $\bullet$ Use class time wisely: fully engage yourself in classroom discussions, asking and answering questions when appropriate.

\noindent  $\bullet$  Seek understanding rather than trying to rely on memorized formulas.

\noindent   $\bullet$ Take advantage of your instructor’s office hours as well as the \href{http://people.virginia.edu/~psb7p/MTCsch.html}{\blue{Mathematics Tutoring Center}}.

\noindent  $\bullet$  It is nearly impossible to understand mathematics without working problems yourself; thus, devoting sufficient time and attention to homework assignments is crucial to success in this course.

\noindent   $\bullet$  Before beginning work on a homework-problem set, think about material discussed in class pertaining to the set---make sure you know and understand the definitions, theorems, concepts, and problem-solving principles emphasized in class. Try to work problems without looking at your notes or the exposition in the text. When you work homework problems without relying on notes, you're re-enforcing your understanding of the principles you reviewed just before beginning work on the problem set. Also, when you take this approach each homework assignment becomes a practice test.
\smallskip



\vskip.15in
\begin{center}
\textbf{Course Content}
\end{center}
  We will cover the following chapters of the course text:\\
Chapter 1: Preliminaries\\
Chapter 2: Functions, Limits, and the Derivative\\
Chapter 3:  Differentiation, skipping 3.4 and 3.7\\
Chapter 4:  Applications of the Derivative\\
Chapter 5:  Exponential and Logarithm Functions (skipping 5.3)\\
Chapter 6:  Integration, up through 6.5.




\vskip.15in
\noindent\textbf{Important Dates}:
\begin{center}
\begin{minipage}{5in}
\begin{flushleft}
	Classes Start \dotfill Wednesday, January 17th\\
	Last day to add a course \dotfill Wednesday, January 31st  \\
	Last day to drop a course \dotfill Thursday, February 1st \\
	Midterm Exam 1 \dotfill  Thursday, February 22, 7--8:30 p.m.\\
	Last day to withdraw from a course: \dotfill Wednesday, March 14th   \\
	Midterm Exam 2  \dotfill Thursday, April 12th, 7--8:30 p.m.\\
	Last day of classes \dotfill Tuesday, May 1st  \\
	Final Exam \dotfill Tuesday, May 8th, 7--10:00 p.m. \\
\end{flushleft}
\end{minipage}
\end{center}
\vskip.15in

I reserve the right to change the syllabus at any time.  (But changes are rare)


\end{document}
